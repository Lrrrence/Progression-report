\chapter{Emerging techniques for online monitoring}\label{ultrasonicmonitoring}

The ideal sensor for jet engine turbine blade and NGV temperature monitoring should be capable of mapping the temperature of a blade or vane in real time with minimum interference to operating conditions. The use of a sensor should not reduce the lifetime of the blade, cause damage to its structure, require large amounts of power, or require regular maintenance, ideally surviving for the lifetime of the engine. Response time should be fast in order to accurately record the changes in temperature during start-up, shut-down, and overshoot events, while resolution and accuracy should be comparable to traditional monitoring methods. 

Ultrasonic guided wave technology may be suitable for this application as it can provide real-time sensing, fast response times, and the ability to re-use the sensor indefinitely. There is the potential for transducers to be kept away from the harsh conditions of the turbine by transmitting a wave through the structure of a blade or vane and analysing the received signal. Measuring temperature in this way reduces the influence of the sensor on the operating condition of the blade. The small footprint of the sensors would not affect the operation of components or disrupt airflow. Advancements in sensor materials for use at high temperatures as well as the associated signal processing may allow for sensors to be installed in harsh environments. The following section considers the suitability of acoustic methods for the temperature monitoring of turbine blades and NGVs.

\section{Ultrasonic structural health monitoring}

Ultrasound is of particular interest for SHM applications as it allows for small sensors, high precision, fast data rates, and can be utilised at frequencies much higher than environmental noise~\cite{Mitra2016}. Traditional ultrasonic NDE utilises A-scans, a measurement of signal amplitude against time, to detect cracks, defects etc. This can be extended to B~\cite{Fatemi1980}, C~\cite{Ruzek2006,Imielinska2004}, or phased-array~\cite{Komura2001} scans to build an image of damage in an area by moving the transducers around and carrying out multiple measurements. This form of evaluation is not particularly suited to turbomachinery applications as the transducers would ideally be permanently installed on the structure. Guided wave SHM is of more interest as constructive interference with surfaces/boundaries allows waves to travel large distances with limited attenuation, which is already utilised for pipe~\cite{Zaghari2013} and rail inspection methods~\cite{Shi2019}. Ultrasound has been proposed as a method of defect detection NDE for aircraft~\cite{Wang2019}, installing transducers in large arrays to allow for guided wave tomography.

When using acoustic waves for SHM and NDE applications either a standalone sensor can be placed in an area of interest, or the structure of interest can directly be used as the sensing medium. In the context of high temperature turbomachinery it would be advantageous to avoid placing sensors into the gas path as they are difficult to interrogate and have the potential to affect operation of the engine components. The structure of turbine blades and NGVs can be used as the sensing medium assuming that an appropriate system of transducers can be implemented.

When considering using a structure as a sensing medium its geometry has an impact on wave propagation. The geometry of a typical turbine blade or NGV is that of a thin plate like structure through which ultrasonic guided waves will propagate. Rayleigh waves (otherwise known as ``Surface Acoustic Waves'' (SAWs)) can be excited at high frequencies (at wavelengths much smaller than material thickness) and are confined to the surface of a material, while Lamb waves (otherwise known as ``Guided Waves'') will propagate if excited at frequencies with wavelengths in the order of material thickness, as they interact with both the top and bottom boundaries of a material. Although Rayleigh waves are non-dispersive they produce large surface motions that are highly sensitive to any discontinuities or defects and are highly affected by surface coatings such as TBCs~\cite{Dransfeld1970}. Lamb waves will propagate through the multi-layered structure of substrate-bonding layer-TBC, where the through-thickness temperature gradient will affect wave propagation, rather than only the temperature at the surface of the TBC. The interaction between both boundaries of a material means that Lamb waves can be excited from the internal surface of an NGV, outside of the gas path. This reduces the influence of the sensor on turbine operation, and reduces the effect of temperature on sensor operation.

\subsection{Temperature compensation techniques} 
In many applications of Lamb wave SHM/NDE the effect of temperature on wave propagation is an undesirable factor that is compensated for using various signal processing techniques. These methods attempt to eliminate the influence of temperature in order to isolate the variable of interest i.e. defects. Temperature compensation or calibration can be carried out by using thermocouples~\cite{2-gajdacsi2014reconstruction}, but this is not ideal for the cases where the external use of a sensor is prohibited. Temperature compensation techniques, such as baseline signal stretch (BSS)~\cite{9R-12-croxford2007strategies, 1R-22-harley2012scale}, optimal baseline selection (OBS)~\cite{1R-24-konstantinidis2006temperature}, hybrid combination of BSS and OBS~\cite{1R-26-clarke2010guided, 1R-27-croxford2010efficient}, combination of OBS and adaptive filter algorithm~\cite{1R-28-wang2014adaptive}, Time of flight ($t_F$) calibration based on a linear relationship between $t_F$ and temperature~\cite{1R-li2018new}, Hilbert transform and the orthogonal matching pursuit algorithm~\cite{1R-20-liu2016baseline}, and temperature compensation for both velocity and phase changes~\cite{9-mariani2019compensation, 9R-16-herdovics2019compensation} have been proposed to reduce residuals between the baseline signal and the current signal. 

BSS method only targets the wave velocity change, neglecting the change in phase and amplitude of the signal. BSS method is restricted for small temperature variations. OBS method can accommodate larger temperature variations, however it requires many baseline signals that have sufficiently low post-subtraction noise levels at different temperatures. Therefore, a combination of BSS and OBS can achieve temperature compensation with a low number of baselines. A new method introduced by Mariani \textit{et al.}~\cite{9-mariani2019compensation} considers amplitude and phase changes as well as wave speed changes. It is shown that the residual between the baseline signal and current signals roughly halved when the two signals were acquired at temperatures 15\si{\degreeCelsius} apart~\cite{9-mariani2019compensation}. Since this method has not been used for measuring the temperature, the sensitivity of the technique has not been analysed. 

\section{Acoustic temperature sensing}

Bulk acoustic waves such as longitudinal and shear waves have been used for temperature sensing for a number of years. An overview of temperature sensing using acoustic waves is given by Lynnworth~\cite{Lynnworth1990}. Davis \textit{et al.}~\cite{Davis2008} experimented with an acoustic temperature sensor during variable frequency microwave curing of a polymer-coated silicon wafer. A sapphire buffer rod was used to separate the wafer from a Zinc Oxide (ZnO) transducer centred at approximately 600 MHz. Time of flight ($t_F$) was measured from the wafer/air interface reflection. Results were comparable to thermocouple measurements from 20\si{\degreeCelsius} to 300\si{\degreeCelsius}, $\pm$2\si{\degreeCelsius}. Takahashi \textit{et al.}~\cite{Takahashi2008} carried out 1D measurements of temperature distribution through a 30 mm thick steel plate in contact with molten aluminium (700\si{\degreeCelsius}). They noted that the system had a faster response time than the thermocouples used for validation of the method. Jia \& Skliar~\cite{Jia2016,Skliar2015} have demonstrated a method for ``UltraSound Measurements of Segmental Temperature Distribution'' (US-MSTD) in solids, utilising reflections along the signal path to estimate temperature distribution. Three different methods of parametrizing the segmental temperature distribution are discussed. The system has been tested in an oxy-fuel combustor at temperatures up to 1100 degrees, with results comparable to thermocouple measurements~\cite{Jia2019}. Jeffrey \textit{et al.}~\cite{Jeffrey2019} demonstrates 2D spatial ultrasonic temperature measurement through a container of wax using 8 0.5 MHz transducers (4 transmitters, 4 receivers), with results comparable to thermocouple measurements. Balasubramaniam \textit{et al.}~\cite{Balasubramaniam1999} developed a temperature measurement system in which an externally cooled buffer rod was used to separate an ultrasound transducer from the hot material under test (molten glass). Changes in time of flight ($t_F$) with temperature were measured from reflections from a notch placed close to the solid-liquid boundary. A calibration procedure carried out from 25\si{\degreeCelsius} to 1200\si{\degreeCelsius} was used to compensate for the thermal gradients of the delay line, and results were compared against a thermocouple  showing a greater than 2\% precision. It was noted that the change in $t_F$ due to temperature was greater than the change due to thermal expansion, leading to a non-linear relationship between temperature and time difference. Measurements were then carried out on molten glass with a resolution of 5\si{\degreeCelsius} (1 ns precision) using a 10 MHz transducer, however the author suggests resolution could be improved to 0.5\si{\degreeCelsius} with faster sampling. Ultrasonic oscillating temperature sensors (UOTSes) are another way to utilise ultrasound for temperature sensing whereby two transducers are setup to transmit through the material of interest in a feedback loop. A number of architectures are described by Hashmi \textit{et al.}~\cite{Hashmi2015, Hashmi2019}, with sensitivities up to 280 Hz/K~\cite{Alzebda2010}.

The uses of guided waves for temperature sensing purposes are limited. However, the fundamental antisymmetric Lamb wave mode, $A_0$, has been used for temperature monitoring of silicon wafers during rapid thermal processing~\cite{Lee1994a,Lee1996}. Quartz pins are used as waveguides, connecting to the wafer through Hertzian contact points. Time of flight ($t_F$) was measured at a rate of 20 Hz from 100\si{\degreeCelsius} to 1000\si{\degreeCelsius} with an accuracy of $\pm$5\si{\degreeCelsius} with this method. A laser excitation system has also been used to measure the temperature of silicon wafers during rapid thermal processing~\cite{Klimek1998}. A broadband excitation pulse is used, and two different methods of signal processing are compared. A cross correlation method between a room temperature baseline signal and those at elevated temperatures, plus a matrix comparison method whereby an unknown signal is compared to a database of signals taken over the whole temperature range of interest. The matrix method was shown to require less signal averaging than the cross-correlation method, with an accuracy in the order of 1\si{\degreeCelsius}.
The current implementations of acoustic temperature sensors demonstrate the potential of this method, but adapting these systems for use on turbine blades and NGVs will be challenging, starting with the method of effectively coupling to their structure. 

\subsection{Measurement devices for the generation and acquisition of guided waves}\label{wavegen}
The choice of transducer for the generation and acquisition of guided waves is limited by a number of factors including: elevated temperatures, space, and power consumption. When transmitting a wave through the structure of a material transducers can either be operated in pulse-echo mode or in a pitch-catch configuration. Pulse-echo mode operates with a single transducer acting as both transmitter and receiver, where the signal reflected from features such as defects or boundaries are analysed. Pitch-catch configurations operate with multiple transducers, some acting as transmitters and some as receivers. Response times are faster in pitch-catch configurations as waves travel directly between transducers rather than requiring a reflection to operate. Systems designed to map defects or damage (tomography) operate in pitch-catch configurations, usually with multiple pairs of transmitters/receivers to allow for higher spatial accuracy. For rotating turbine blades a transducer could be implemented in a pulse-echo configuration, if a transducer could not be housed at the rotating tip of the blade. Transducers could be operated in a pitch-catch configuration on NGVs as transducers could be placed on either side of the static vane. 

Piezoelectric based transducers are amongst the most common methods of generating and acquiring acoustic waves. They can be utilised in a number of different ways for either dedicated sensors or for transmitting a wave through a particular structure of interest. Many dedicated sensors are based on the use of interdigital transducers (IDTs) operated as either delay lines or resonators, and are often referred to as surface acoustic wave (SAW) devices. These sensors can be used for high temperature sensing and can be interrogated wirelessly~\cite{PereiraDaCunha2011,Dong2019}, however they would be difficult to implement on the structure of a turbine blade or NGV, while only providing a single point measurement. 

\begin{figure}[!htbp]
    \centering
    \includegraphics[width=.8\textwidth]{./figures/SAW delay line.png}
    \caption{A SAW delay line~\cite{Ida2014}}\label{fig:sawdelay}
\end{figure}

\begin{figure}[!htbp]
    \centering
    \includegraphics[width=.8\textwidth]{./figures/SAW resonator.png}
    \caption{A SAW resonator~\cite{Ida2014}}\label{fig:sawresonator}
\end{figure}

In order to transmit a wave through a structure using piezoelectrics there are a number of options, namely wafers and wedges. These sensors are reliant on an effective bond between transducer and substrate, which is difficult to a achieve at high temperatures.

Wedges (Figure~\ref{fig:wedge}) can be used to generate surface acoustic waves based on Snell’s law~\cite{Zhang2017a}, where a longitudinal wave produced by a piezoelectric transducer is transmitted through an angled wedge (typically made from a material with a slow longitudinal wave speed relative to the substrate, such as acrylic) into a substrate material where wave refraction takes place. Transmission of shear or surface acoustic waves are dependent on the material properties of the wedge and the substrate, and the wedge angle. A large benefit of this method is the ability to excite single Lamb wave modes in one direction~\cite{Khalili2016}. Unfortunately a liquid couplant is required to form an effective bond between wedge and substrate, which is unlikely to be suitable for permanent installation at high temperatures.

Piezoelectric Wafer Active Sensors (PWAS) are being used extensively for SHM applications and have been shown to withstand exposure to extreme environments~\cite{Mei2019}. They are non-resonant wide-band devices~\cite{Giurgiutiu2003a} however they can be used for generation of single Lamb wave modes with careful geometry selection~\cite{Ren2017}. PWAS are small, inexpensive, and minimally invasive~\cite{Giurgiutiu2003a}, making them potentially suitable for installation on turbine blades and NGVs if a suitable bonding method and piezoelectric material can be found.

\begin{figure}[!htbp]
    \centering
    \includegraphics[width=.8\textwidth]{./figures/wedge.eps}
    \caption{A typical wedge transducer}\label{fig:wedge}
\end{figure}

One promising solution to the bonding problem is to use a waveguide to separate the transducer from the extreme environment of the turbine. This method is especially suited to guided waves as they are known to travel large distances with little attenuation, however the strong reflections from the boundaries of the material can introduce dispersion, and material discontinuities should be considered~\cite{Parks2013}. One of the main challenges of this method is how to couple the waveguide to the structure of interest, as liquid couplants (as used for wedge coupling) are not suitable for use in high temperature environments, and simple welding is likely to introduce defects causing additional reflections and discontinuities to occur. It has been shown that clamping a waveguide to a structure can be effective even at relatively high temperatures (700\si{\degreeCelsius})~\cite{Cegla2011a}, although this method may not be suitable for installation on a turbine blade or NGV.
\\
\begin{figure}[!htbp]
    \centering
    \includegraphics[width=.8\textwidth]{./figures/hertzian.eps}
    \caption{An example Hertzian contact transducer}\label{fig:hertzian}
\end{figure}
\\
A system of Hertzian contact points (Figure~\ref{fig:hertzian}) may be more appropriate as only a small point would need to be in contact with the surface of interest. The contact size of the point determines the aperture size of the source, which can be considered as a point source up to very high frequencies, allowing for accurate measurement of absolute velocity~\cite{LeventDegertekin1997}. This method of coupling has been used to measure the mechanical properties of carbon fibre reinforced plastics (CFRP) using measured phase velocities of the $A_0$ and $S_0$ Lamb wave modes~\cite{Grimberg2010}. Application force of the rods can be controlled with springs. An example of a welded waveguide system installed on a turbine vane is given by Willsch~\cite{Willsch2004a}.

\begin{figure}[!htbp]
    \centering
    \includegraphics[width=.8\textwidth]{./figures/willschwaveguide.png}
    \caption{Willsch's welded waveguide~\cite{Willsch2004a}.}\label{fig:willsch}
\end{figure}

Other options for excitation/acquisition of acoustic waves include electromagnetic acoustic transducers (EMATs)~\cite{Khalili2018c} or lasers. Both of these methods could allow for contactless operation, however EMATs are quite inefficient, requiring large amounts of power to produce a signal of adequate amplitude. The use of lasers would require optical access to the surface of interest as well as the installation of a patch to protect the surface of the substrate from laser ablation~\cite{Kim2019a}.

Although the use of piezoelectric transducers is likely to be the most appropriate method of exciting acoustic waves for this application, temperature has an effect on their operation~\cite{4-jiang2014high, 12-fachberger2004properties}. The resonant frequency of a piezoelectric transducer reduces as temperature elevates~\cite{4-jiang2014high} and if the transducer operates at the resonance ringing may be seen at some temperatures~\cite{9-mariani2019compensation}. Filtering methods have been proposed to compensate for the transducer transfer function at different frequencies~\cite{ 9R-18-hurst2015real}. Temperature variations affect the bonding stiffness at the interface between the transducer and the structure. This results in frequency response changes which can affect both amplitude and the phase of the signal~\cite{9R-14-ha2010adhesive}. A common problem with all methods of excitation using piezoelectrics is their reduction in sensitivity with increasing temperature, which makes the choice of piezoelectric material vitally important.

\subsection{Piezoelectric selection for high temperature sensing} 
Piezoelectric materials for high temperature sensors have been compared by a number of authors~\cite{Turner1994,Tressler1998,Tittmann2013,Parks2013,Jiang2013b,Schulz2003,Zhang2011,Zhang2018}. Aluminium Nitride (AlN), Langasite (LGS), and Rare Earth Calcium Oxyborate Single Crystals (ReCOB), specifically Yttrium Calcium Oxyborate Single Crystals (YCOB), can be used for high temperature sensing with piezoelectrics~\cite{stevenson2015piezoelectric}.
YCOB has the highest operating temperature besides having a high resistivity at high temperatures. This signifies that YCOB will be able to operate at higher temperature as less heat is dissipated with a lower current. YCOB has relative small degradation in sensitivity with increasing temperatures of up to 1000\si{\degreeCelsius}, together with no significant phase change up to temperatures of 1500\si{\degreeCelsius}, which makes it ideal for high accuracy temperature measurements~\cite{zu2016high}. YCOB also has a linearly decreasing resistance with temperature, and a close to linearly decreasing resonance frequency with temperature, which can both be used for temperature measurement~\cite{zhang2008characterization}.

Aluminium Nitride (AlN) also exhibits a number of very promising properties necessary for high temperature sensing such as high electrical resistivity, temperature independence of electromechanical properties, and high thermal resistivity of the elastic, dielectric, and piezoelectric properties~\cite{Kim2015}. The use of an AlN sensor up to 800\si{\degreeCelsius} has been demonstrated for detection of laser generated Lamb waves in thin steel plates by Kim~\cite{Kim2018a}. Unfortunately high-quality AlN single crystals are difficult to grow, showing a wide range of resistivity that greatly affects their suitability as ultrasound transducers~\cite{Tittmann2013a}.

\section{Introducing Lamb waves for temperature sensing}\label{lambwavesensing}

The geometry of turbine blades and nozzle guide vanes allows Lamb waves to propagate through their structure at ultrasound frequencies. Lamb waves are a type of elastic wave present in thin plates when wavelength is in the order of thickness. They are guided by the upper and lower boundaries of a material allowing for continuous wave propagation (Figure~\ref{fig:lambwaves})~\cite{Rose2014}. The bulk acoustic waves discussed previously are non-dispersive, i.e their wave velocities are constant with frequency, whereas Lamb waves are dispersive and multi-modal which makes their analysis complex, especially when their are other factors such as changing temperatures are involved. The lowest order modes, the fundamental antisymmetric mode $A_0$, and the fundamental symmetric mode $S_0$ (Figure~\ref{fig:incophasedisp}), are the most commonly used modes as they are relatively non-dispersive and comparatively easy to generate in comparison to the higher order modes ($A_1$, $S_1$, etc.). Lower order Lamb waves are used extensively for NDE and SHM applications and an overview of their uses for damage identification is provided by Su~\cite{Su2009a}. Lamb waves have both phase and group velocities, the phase velocity relating to the local speed with which phase of the wave changes, and a group velocity
which describes the overall speed of energy transport through the propagating wave. Phase velocity is generally higher than the group velocity. Time of flight ($t_F$) measurements of Lamb waves give the group velocity, while special phase comparison techniques are needed to measure the phase velocity~\cite{Cheeke2000a}. 

\begin{figure}[!htbp]
    \centering
    \includegraphics[width=.8\textwidth]{./figures/lambwaves2.eps}
    \caption{Particle displacement of symmetric and antisymmetric Lamb wave modes in a plate.}\label{fig:lambwaves}
\end{figure}

\begin{figure}[!htbp]
    \centering
    \includegraphics[width=.8\textwidth]{./figures/inconelyoungsdensitypoisson.eps}
    \caption{Temperature dependent Young's modulus, density, and calculated Poisson's ratio for Inconel 718.}\label{fig:dpe}
\end{figure}

Lamb waves are affected by three main factors due to changes in temperature: thermal expansion, variations in Young's modulus, and transducer response (including bonding). The effect of temperature on density, Poisson's ratio, and Young's modulus, is shown in Figure~\ref{fig:dpe} for the superalloy Inconel 718~\cite{Abdullaev2019,SpecialMetals2007}. Nickel based superalloys such as these are commonly used for aerospace components that are exposed to high temperatures. A change in temperature has the greatest effect on Young's modulus in comparison to changes in thermal expansion (density and Poisson's ratio)~\cite{Abbas2020a}. These changes affect guided wave velocity~\cite{9R-10-weaver2000temperature}, an example of which is given in figure~\ref{fig:alutempshift} for the $S_0$ mode when excited in an aluminium plate using wedge transducers with a 5 cycle tone burst, where it can be seen that an increase in temperature from 20\si{\degreeCelsius} to 60\si{\degreeCelsius} reduces wave velocity and signal amplitude. A number of authors have investigated the temperature dependence of Lamb waves~\cite{Dodson2013,Marzani2012,Croxford2007a,Abbas2020a,Moll2019} however their studies are limited to relatively low temperatures. To better understand the uses of Lamb waves their frequency spectrum can be split into three areas:
\begin{itemize}
    \item Low frequency region – Contains the lowest two Lamb wave modes ($A_0$ \& $S_0$). It is possible to selectively excite either mode as they have very different wave velocities, leading to large differences in excitation angle when using a wedge transducer for example.
    \item Mid-frequency region – Contains many dispersive modes with similar wave speeds making it difficult to excite specific modes without exciting others, leading the production of complex waveforms that are difficult to analyse. 
    \item High frequency region – The modes become less dispersive and converge to similar wave speeds, leading to them travelling as a single packet. Group velocity can be measured for the packet. The $A_0$ \& $S_0$ modes begin to act like a Rayleigh wave as their velocities converge.
\end{itemize}

\begin{figure}[!htbp]
    \centering
    \includegraphics[width=.8\textwidth]{./figures/alutempshift2560.eps}
    \caption{The effect of temperature on the $S_0$ mode in an aluminium plate.}\label{fig:alutempshift}
\end{figure}

Figure~\ref{fig:incophasedisp} shows phase velocity curves for the symmetric and anti-symmetric Lamb wave modes generated by \href{https://www.dlr.de/zlp/en/desktopdefault.aspx/tabid-14332/24874_read-61142/#/gallery/33485}{The Dispersion Calculator}~\cite{Huber} for the superalloy Inconel 718. Dispersion curves such as these can be be generated based on material properties allowing areas of interest to be identified. 

\begin{figure}[!htbp]
    \centering
    \includegraphics[width=.8\textwidth]{./figures/phasedispersion.eps}
    \caption{Lamb wave phase velocity dispersion curves for Inconel 718.}\label{fig:incophasedisp}
\end{figure}

As an example of Lamb wave temperature dependence Figure~\ref{fig:groupshift} shows the shift in group velocity dispersion curves for the superalloy Inconel 718 from 21\si{\degreeCelsius} to 1093\si{\degreeCelsius}. An increase in temperature causes a reduction in wave velocity (shifting down) and a reduction in frequency (shifting left). The temperature dependence at a particular frequency can also be calculated, which is useful when determining a suitable excitation frequency. Careful selection of excitation frequency can allow for high temperature sensitivity depending on the mode excited and the dispersiveness of the region. As an example figure~\ref{fig:grouptempdependance} shows the temperature dependence of the $A_0$ \& $S_0$ modes at a frequency-thickness product of 1~MHz-mm. The change in wave velocity at this frequency-thickness product is non-linear because of two factors, a non-linear change in Young's modulus with increasing temperature, and the chosen frequency of 1~MHz falling into a more dispersive region as temperature increases, particularly for the $S_0$ mode. Average temperature sensitivity for the $A_0$ mode is \mbox{$-$0.898 m s$^{-1}$\si{\degreeCelsius}$^{-1}$} and \mbox{$-$1.868 m s$^{-1}$\si{\degreeCelsius}$^{-1}$} for the $S_0$ mode. It can be seen from Figure~\ref{fig:groupshift} that the $A_0$ mode has a relatively linear reduction in wave velocity with increasing temperature regardless of frequency, whereas the other modes have highly dispersive regions (steep slopes) that reduce in frequency (shift to the left) with increasing temperature. The high sensitivity to changes in temperature in these regions have great potential for temperature monitoring applications. Resolution is dependent on sampling frequency and the choice of time of flight ($t_F$) measurement method~\cite{Espinosa2018,Svilainis2013}. If a sampling rate of 2.5 GHz is used on the example given above over a distance of 10 mm (rough distance to first line of cooling holes) a theoretical velocity resolution of 0.4 m s$^{-1}$ for the $S_0$ mode and 0.2 m s$^{-1}$ for the $A_0$ mode can be achieved at 1093\si{\degreeCelsius}, which would allow for a temperature resolution of \textless{1\si{\degreeCelsius}} at 1~MHz. This would require measurements of $t_F$ at sub-wavelength resolution which, although challenging, can be achieved with cross-correlation methods~\cite{Jia2019a,Khyam2017}. Linear interpolation of cross-correlation methods can be used to increase resolution without increasing sampling rate~\cite{Costa-Junior2018}.

\begin{figure}[!htbp]
    \centering
    \includegraphics[width=.8\textwidth]{./figures/grouptemp.eps}
    \caption{$A_0$, $S_0$, $A_1$, and $S_1$ group velocity dispersion curve shift with temperature from 21\si{\degreeCelsius} to 1093\si{\degreeCelsius} for Inconel 718.}\label{fig:groupshift}
\end{figure}

\begin{figure}[!htbp]
    \centering
    \includegraphics[width=.8\textwidth]{./figures/grouptemp1mhztemp.eps}
    \caption{$A_0$ \& $S_0$ group velocity change with temperature from 21\si{\degreeCelsius} to 1093\si{\degreeCelsius} at 1~MHz for Inconel 718.}\label{fig:grouptempdependance}
\end{figure}

Although the non-dispersive nature of lower order modes ($A_0$ \& $S_0$) makes them relatively easy to analyse, it would be advantageous to operate at higher frequencies as phase shifts are easier to detect when wavelengths are shorter, which leads to improved sensing resolution and accuracy. As the wave speeds of the $A_0$ \& $S_0$ modes converge (around 10 MHz-mm for Inconel 718) they behave as a Rayleigh wave, which limits their use for this application as discussed previously. Higher order modes such as $A_1$ \& $S_1$ are more difficult to selectively excite as their phase/group velocities are similar, which leads to the formation of complex waveforms that are highly dispersive. However, as frequency-thickness product is increased further the excitability of higher order modes reduces dramatically which can allow less dispersive regions of lower order modes to be excited~\cite{Wilcox2004} (see high frequency region of figure~\ref{fig:incophasedisp}). Wilcox \textit{et al.}~\cite{Wilcox2003} have presented a method of reducing the effect of dispersion on a transmitted signal if prior knowledge of dispersion curves are known. This relaxes the need to excite only a single mode and simplifies the analysis process. In the case of turbine blades and NGVs, the propagation distance is relatively short so effect of dispersion is likely to be low, however a change in temperature will cause different regions of dispersion curves to be excited which will change the shape of a transmitted wave packet.

Jayaraman \textit{et al.}~\cite{Jayaraman2009a} have presented the existence of ``Higher Order Mode Cluster Guided Waves'' (HOMC-GW), a non-dispersive region found at high frequency-thickness products in which the various modes all have similar group velocities. This causes them to move as a single envelope, which can be treated like a single non-dispersive mode.  A number of aspects of HOMCs have been investigated including: their use for pipe inspections~\cite{Swaminathan2011, Chandrasekaran2010a, Balasubramaniam2008}, their interaction with weld pads~\cite{Verma2011}, and their interaction with notch-like defects in plates~\cite{SriHarshaReddy2017a}. Khalili \& Cawley~\cite{Khalili2016} carried out an investigation into exciting singular higher order modes which found that the HOMC described by Jayaraman was likely to be single mode dominating a cluster ($A_1$ around 20 MHz-mm). The use of this higher order mode cluster for temperature sensing is of particular interest for turbine blade applications and further investigation is required.

\section{Conclusion}

A review of the current methods of temperature sensing for turbine blades and NGVs in jet engines has been carried out. Offline systems such as thermal paints and thermal history sensors are well established, but provide limited data in comparison to online systems. When used for the validation of thermal models online systems can provide considerably more information, covering temperature changes through start-up to shut-down of an engine, as well as recording over shoot events. There are a number of online systems available including thin film thermocouples (TFTCs), thermographic phosphors, and pyrometers. Thermocouples need to be embedded into the structure of a component and require a wire connection which is a significant point of failure, while only providing point measurements unless installed in dense arrays. Pyrometers can provide temperature maps without installing sensors onto the surface of a component, but optical access is required, and environmental factors have a significant impact on their accuracy. Thermographic phosphors require optical access to components for both excitation and analysis, as well as direct application of a phosphor to the surface. This makes sensors difficult to implement for condition monitoring applications because of space constraints, especially in jet engines.

The ideal sensor for this application would operate outside of the gas path without interfering with operation of components, while still providing a high degree of accuracy with fast response times. Acoustic methods offer a potential advantage in their low power operation, small footprint, and temperature mapping potential. Further investigation is required to fully understand wave propagation through the complex geometries of turbine blades and NGVs. This includes coating materials and thicknesses, their temperature dependency, and their degradation (for example pitting and surface microcracking). Changes in surface characteristics can significantly alter the attenuation of guided waves and cracks/defects would cause additional reflections to occur. Residual stresses relating to high temperature gradients will also modify guided wave behaviour locally. Further investigation is required to determine the best method of attaching transducers to a blade or vane. Using waveguides to reduce the impact of temperature on the operation of transducers is likely to be the most appropriate option although there are a number of piezoelectric materials (most notably YCOB and AlN) that would be suitable for use at extremely high temperatures, allowing transducers to be mounted directly onto the blade housing.

The literature covering the temperature dependence of guided waves is limited to relatively low temperatures and generally only considers the lowest order fundamental Lamb wave modes ($A_0$ \& $S_0$). To achieve a temperature resolution that is comparable with traditional sensors the frequency of operation needs to be high in order to accurately detect a phase shift with changing temperature, which highlights the potential of higher order modes. Relative measurement of wave velocity is the most common method of temperature sensing using acoustic waves as in most cases (using non-dispersive waves) the change with temperature is linear. When utilising Lamb waves for this purpose careful excitation of single modes will reduce the complexity in signal analysis however this may not be possible with the limited range of transducers and mounting points available. Comparisons with baseline signals may be more appropriate which are already utilised in a number of different temperature compensation techniques. In order to map temperature distribution across a blade the complex series of reflections from cooling holes and boundaries need to be utilised. Reflections can cause mode conversion to take place that has the potential to add additional complexity to the system. This presents a signal processing challenge that has not been previously considered for this application. Further research is planned to evaluate the potential of a guided wave based temperature sensing system for turbine blades and NGVs. The temperature dependence of Lamb wave modes will be theoretically predicted and measured experimentally up to high temperatures. The results of this investigation can be used to determine the most suitable mode for temperature sensing. The interaction between guided waves and cooling holes will be investigated for the purposes of mapping the temperature across a blade or vane. Methods of coupling transducers to the structure of a blade will be considered based on the ability to excite the mode of interest and survive in the high temperature environment of a gas turbine.

\section{Other applications}

A Lamb wave based temperature monitoring system has potential for other applications. Batteries and battery packs are often arranged as thin cells, where the temperature of the cell gives an indication of cell health. Lamb waves would allow the monitoring of the cross-sectional temperature of a cell with the use of small piezoelectric sensors, giving an advantage over point-based systems such as thermocouples.