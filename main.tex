\documentclass{report}
\usepackage[utf8]{inputenc}
\usepackage{gensymb}
\usepackage{color,soul}
\usepackage{graphicx}
\usepackage{xcolor}
\usepackage{amssymb}
\usepackage{tabu}
\usepackage{tabulary}
\usepackage{booktabs}
\usepackage{moreverb,url}
\usepackage[colorlinks,bookmarksopen,bookmarksnumbered,citecolor=red,urlcolor=blue]{hyperref}
\usepackage{textcomp}
\usepackage{siunitx}
\usepackage{tabularx}
\usepackage{amsmath}
\usepackage[raggedright]{titlesec}
% stops headings from hyphenating
\usepackage{geometry} % to change margins
\usepackage{ragged2e} % provides \RaggedLeft
\usepackage{pdflscape}
\usepackage{pgfgantt}
\usepackage{standalone}
\usepackage{ragged2e} 
\usepackage{typearea} 
\usepackage{parskip}
\usepackage{pdfpages}
\usepackage[margin=10pt,font=small,labelfont=bf,
labelsep=endash]{caption}

\definecolor{barblue}{RGB}{153,204,254}
\definecolor{groupblue}{RGB}{51,102,254}
\definecolor{linkred}{RGB}{165,0,33} 

\geometry{a4paper,textwidth=15cm, textheight=21.3cm, marginratio={1:1,5:7}}

\begin{document}

\begin{titlepage}
    \begin{center}
        
            
        \Huge
        \textbf{Confirmation Review}
            
        \vspace{3cm}
        \Large
        Temperature monitoring of nozzle guide vanes using ultrasonic guided waves
     
            
        \vspace{3cm}
            
        \textbf{Lawrence Yule}
            
        \vspace{3cm}
            
        \underline{Supervisors} \\
        \vspace{0.5cm}
        Nick Harris, Bahareh Zaghari, and Martyn Hill
        
        \vfill
        \today
        \vspace{1cm}
            
        
        \large
        Electronics and Computer Science\\
        Smart Electronic Materials and Systems Research Group\\
        University of Southampton\\
   
    \end{center}
\end{titlepage}

\begin{abstract}

This study will investigate the use of ultrasonic guided waves (Lamb waves) for the temperature monitoring of nozzle guide vanes (NGVs). These components are found in the turbine section of jet engines and are operated at extremely high temperatures. A literature review has been carried out that covers the existing temperature monitoring systems for turbine blades and nozzle guide vanes. Both offline and online methods are presented and their advantages and disadvantages are examined. The use of offline systems is well established but their online equivalents are difficult to implement because of the limited access to components. There is the need for an improved sensor that is capable of mapping temperature in real time with minimum interference to the operating conditions of the engine, allowing operating temperatures to be increased to the limits of the components and maximising efficiency. Acoustic monitoring techniques are already used for a large number of structural health monitoring (SHM) applications and have the potential to be adapted for use in temperature monitoring for turbine blades and NGVs. High temperatures severely affect the response of ultrasonic transducers. However, waveguides and buffer rods can be used to distance transducers from extreme conditions, while piezoelectric materials such as YCOB and AlN have been developed for use at high temperatures. The geometry of turbine blades and NGVs allows Lamb waves to propagate through their structure, and the presence of numerous cooling holes will produce acoustic reflections that have the potential to be utilised for temperature mapping.

A number of studies are underway to investigate the effects of the physical environment on wave propagation, and to determine the Lamb wave mode best suited to the application. The generation of dispersion curves from material properties allows theoretical temperature sensitivity to be determined, which has been verified experimentally. A test system has been developed to target modes of interest, and analyse the effect of temperature (and other factors) on wave propagation. A COMSOL model has also been developed that will be used in future studies to investigate the effect of cooling holes, surface coatings, and curved surfaces on wave propagation. The model will also be used to investigate different sensor configurations that can be implemented on an NGV. 

\end{abstract}

{
  \hypersetup{hidelinks}
  \tableofcontents
  \listoffigures
  \begingroup
  \let\clearpage\relax
  \listoftables
  \endgroup
}

\chapter{Introduction}
Nozzle guide vanes (NGVs) are static components found in the turbine section of a jet engine (Figure~\ref{fig:crossngv}). Temperature monitoring of these components is important for a number of reasons: identifying potential failures before they occur, evaluating the need for maintenance, investigating ways of improving engine efficiency, and reducing fuel consumption. Online temperature monitoring of NGVs is difficult to achieve with currently available technologies (see Chapter~\ref{literev}). Acoustic based monitoring methods offer a potential alternative. The velocity of acoustic waves is affected by temperature (among other properties) and can be used for temperature monitoring if a suitable sensor system can be developed and implemented. Transducers can be used to transmit acoustic waves through a material and analyse the received signal. The transducers can be placed away from harsh conditions (e.g.\ turbine gas paths) which would otherwise limit their operation. In the case of nozzle guide vanes (NGVs) their geometry allows for the propagation of Lamb waves at ultrasound frequencies. These waves are dispersive and many distinct modes can be present, which makes their analysis difficult. Despite this problem they can provide high temperature sensitivities (mode dependant) and can travel large distances with limited attenuation. Generation of these waves is possible with a number of different transducer configurations, but the choice is limited by the harsh conditions of the turbine and by space and power constraints. Two potential solutions are the use of waveguides to further distance transducers from the harsh environment, or the use of Piezoelectric Wafer Active Sensors (PWAS) which can be used at high temperatures with careful piezoelectric material selection.
\newpage

The study can be classified into five sections:
\begin{itemize}
    \item Chapter~\ref{literev} -- A literature review of the currently available temperature monitoring methods for nozzle guide vanes, both offline and online.
    \item Chapter~\ref{ultrasonicmonitoring} -- An investigation into the potential for guided wave based temperature monitoring systems.
    \item Chapter~\ref{theory} -- Theoretical analysis of Lamb wave temperature sensitivity.
    \item Chapter~\ref{simulations} -- COMSOL modelling of a temperature monitoring system.
    \item Chapter~\ref{experiments} -- Experimental investigations into Lamb wave temperature sensitivity, and the factors affecting wave propagation.
\end{itemize}

Theoretical evaluation of the temperature sensitivity of Lamb wave modes highlights the potential of the method. The sensitivities of the $S_0$, $A_1$, and $S_1$ modes for a 1 mm think Aluminium plate have been extracted from dispersion curves generated from material properties. The prediction indicates that group velocity will reduce with increasing temperature. An experimental test system has been developed to validate the theoretical predictions. Two variable angle wedge transducers have been used to target the modes of interest, and a measurement of time of flight between transducers has been used to calculate the group velocity. Results are in good agreement with theoretical predictions, showing that a time of flight based measurement system is capable of monitoring a change in temperature effectively.

A COMSOL model has been developed to investigate the effect of other factors on wave propagation. This includes curved surfaces, surface coatings, and cooling holes. The presence of cooling holes is likely to have varying effects on different Lamb wave modes, and the acoustic reflections produced from interaction with the holes raises the possibility of monitoring temperature at a number of locations. The results of this study are likely to determine the most suitable Lamb wave mode (or group of modes) for temperature monitoring of nozzle guide vanes. The effect of holes on wave propagation will also be tested experimentally, to validate the model. Depending on the outcome of the study a signal processing system may be developed to allow for the monitoring of temperature from reflected waves.

For installation on a nozzle guide vane the sensor system is required to operate at extremely high temperatures. A COMSOL model will be developed to test a system that can be used in this environment. A potential candidate is to use waveguides to distance transducers from the high temperatures, while reducing the difficulty in bonding to the structure by using Hertzian contact points. The model can be used to investigate the suitability of this method.

A plan and Gantt chart detailing the aims of the project from 18 months onwards can be found in Section~\ref{plan}.

\newpage
\vspace*{\fill}

\begin{figure}[!htbp]
    \centering
    \includegraphics[width=\textwidth]{./figures/turbine plus ngv.png}
    \caption{Jet engine cross-section with NGV.}\label{fig:crossngv}
\end{figure}

\vspace{\fill}

\newpage

\section{Research objectives}

The overarching research objective is to determine if a Lamb wave based temperature monitoring system is suitable for use on gas turbine nozzle guide vanes. Ideally a system will be capable of monitoring temperature with a resolution, accuracy, and response time, comparable with traditional temperature monitoring systems. With this in mind there are a number of points to be considered, including sensor configuration, wave propagation, environmental impact, and a signal processing. 

One key objective is to understand the effects of the physical environment on wave propagation in an NGV-like structure. This includes evaluating the effects of the propagation medium (material properties, curved surfaces, cooling holes, and surface coatings), and environmental conditions (temperature, acoustic noise, and gas flow) on wave propagation. The conditions listed previously will have differing effects on different Lamb wave modes, which leads to an additional objective of identifying the most appropriate mode (or mode group) for temperature monitoring in this environment. In order to evaluate these factors a test system will be developed that can transmit/receive Lamb waves in an NGV-like structure, which will also allow investigation into the signal processing requirements of a real system. 

Another objective is to identify a suitable sensor configuration that can survive in high temperatures, and operate under the restrictions of the environment. This includes low power operation, strict space constraints, and the ability to last for the lifetime of turbine with minimal servicing. Additionally, the type of sensor and the interface between the sensor and the NGV structure will impact the types of waves that can be generated, which further complicates the selection of suitable sensor configurations.

A requirement for the monitoring of NGVs is the ability to monitor temperature spatially, in order to identify hot spots or likely points of failure. An objective of this study is to determine if the monitoring of temperature at multiple locations is possible, through the acoustic reflections created by wave interaction with cooling holes.

These objectives raise the following research questions:

\subsection{Research questions}
\begin{enumerate}
    \item How can the temperature dependence of Lamb waves be utilised for the temperature monitoring of NGVs?\label{itm:1}
    \item How can the theoretical temperature sensitivity of individual Lamb wave modes be verified experimentally?\label{itm:2}
    \item What is the effect of the physical environment on Lamb wave propagation in NGVs?\label{itm:3}
\begin{itemize}
    \item High temperatures
    \item Cooling holes
    \item Curved surfaces
\end{itemize}
    \item Which of the Lamb wave modes (or group of modes) is most appropriate for temperature monitoring of NGVs?\label{itm:4}
    \item To what extent can acoustic reflections from cooling holes be used to monitor temperature at a number of locations across the structure of an NGV?\label{itm:5}
    \item What is the most suitable transducer configuration for exciting Lamb waves in NGVs?\label{itm:6}
\end{enumerate}

Upon completion of the PhD the research questions listed above will have been addressed, which will lead to a conclusion on how a guided wave based temperature monitoring system can be implemented on a nozzle guide vane. The effect of the physical environment on wave propagation will be known, and the most suitable modal region of temperature monitoring will have been identified. The extent to which the wave interaction with cooling holes can be used to monitor temperature at a number of locations will be known. A measurement system that can theoretically operate at high temperatures will have been identified. The outcome of these studies can form the basis of future investigations into the design of a measurement system suitable for use at high temperatures, and the testing of a system on an installed NGV. 

\section{Research output}
\begin{itemize}
    \item Student poster ``Temperature monitoring of nozzle guide vanes (NGVs) using ultrasonic guided waves'' presented at \href{https://event.asme.org/Turbo-Expo}{ASME Turbo Expo 2020.} Virtual conference 21--25 September 2020.
    \item Journal review paper ``Surface temperature condition monitoring methods for aerospace turbomachinery: exploring the use of ultrasonic guided waves''~\cite{Yule2021} published to \href{https://iopscience.iop.org/journal/0957-0233}{IOP Measurement, Science, and Technology.}
    \item Conference paper ``Towards in-flight temperature monitoring for nozzle guide vanes using ultrasonic guided waves'' accepted for \href{https://www.aiaa.org/propulsionenergy}{AIAA Propulsion Energy}, 9--11 August 2021.
    \item Journal paper on the COMSOL modelling of a guided wave sensor system for harsh environments to be submitted to \href{https://www.mdpi.com/journal/sensors}{MDPI Sensors} in the near future.
\end{itemize}

\chapter{Nozzle guide vane (NGV) temperature monitoring}\label{literev}
This literature review considers surface temperature monitoring methods for turbomachinery applications, focusing on jet engine turbine blades and nozzle guide vanes (NGVs).

Carbon dioxide emission is the largest contributor from aviation to global warming, comprising 70\% of aircraft engine emissions~\cite{administrator2015aviation}. A small improvement in the design of a jet engine will result in a sizable reduction in pollution, as well as in fuel and engine costs~\cite{zhang2019number,Zaghari2020}. The results of temperature monitoring can inform the research and development process to improve the operation and efficiency of components~\cite{Zaghari2017}. More accurate monitoring in terms of absolute temperature, spatial resolution, and time history allows for further development, maximising the effectiveness of components. Current and past methods focus on testing parts in controlled environments before installation in an engine, with the majority of methods requiring the blade or vane to be removed from the engine for analysis after a test has taken place. This can be improved with online methods that allow for monitoring during normal engine operation. Online methods provide considerably more data for analysis during start-up and cool down of the engine rather than only providing a peak result. The currently available online methods can only provide point measurements (in the form of thermocouples) or require optical access (pyrometry \& thermographic phosphors) which is difficult to achieve in the cramped conditions of a turbine. These online systems have been utilised in test rigs and gas turbines but have not been adapted for use on in-service jet engines because of space, weight, and power constraints. An improvement to these sensors would greatly benefit gas turbine manufacturers, allowing for further development in turbine blade and NGV design by better understanding their limits. If the uncertainty of temperature measurements can be reduced then engine efficiency can be improved by operating components closer to their ideal conditions~\cite{Kerr2002}.

An increase in turbine inlet temperature elevates the risk of blade failure from blade creep or oxidation, the extent to which is determined by the exposure time at raised temperatures~\cite{Becker1994}. Blades and vanes require constant cooling with film cooling techniques embedded into their structure to avoid corrosion and melting~\cite{Zhou2016}, with gas temperatures up to 1000\si{\degreeCelsius} using uncooled blades and around 1800\si{\degreeCelsius} with cooled blades~\cite{Dixon2013}. The air used for film cooling is drawn from the main gas path, reducing efficiency while increasing NO\textsubscript{X} emissions~\cite{Heyes2004}. As emission limits become stricter the use of film cooling may have to be reduced, emphasising the importance of temperature monitoring to allow components to be operated closer to their thermal limits. 
Components such as turbine blades and NGVs are coated with ceramic based thermal barrier coatings (TBCs) to allow gas stream temperatures to be further increased. These coatings are normally multi-layered, utilising yttrium-stabilised-zirconia (YSZ; Y$_2$O$_3$–ZrO$_2$) as the insulating material, a thermally grown oxide layer (TGO), and a bonding layer to attach to the substrate~\cite{Thakare2020}. The bonding layer is considered to be the most likely point of failure from exposure to excessively high temperatures~\cite{Feist2001}, monitoring of this area is therefore vital for maintaining blade health. 
Reviews of condition monitoring/non-destructive evaluation (NDE)/structural health monitoring (SHM) for all aspects of gas turbines are provided by Abdelrhman~\cite{Abdelrhman2014} and Mevissen~\cite{Mevissen2019}, however there are a limited number of studies that focus on developments in online surface temperature monitoring for jet engines, which is covered in the first half of this literature review.

\begin{figure}[!htbp]
    \centering
    \includegraphics[width=.8\textwidth]{./figures/Turbinecrosssection.eps}
    \caption{Cross section of a typical turbine blade with TBC applied~\cite{AbouNada2016}}\label{fig:turbineblade}
\end{figure}

Acoustic methods are a potential alternative that have already been utilised for a number of other SHM and NDE applications. Ultrasonic guided waves in particular offer many benefits including low costs, high accuracy, and fast response times if utilised correctly. Acoustic waves transmitted through the structure of a blade or vane would not interfere with their operation, and temperature mapping could be achieved by utilising the acoustic reflections from the large number of cooling holes found across the structure. The difficulties of this method are in managing the impact of high temperatures on electronic components~\cite{Zaghari2017a} and sensor operation, which requires appropriate material selection and signal processing techniques. There are a number of different options for the generation and acquisition of waves, methods of coupling to the structure, and temperature derivation, all of which need to be investigated further to determine the most suitable method. The second half of this literature review discusses the potential of acoustic methods and how they can be implemented for use on turbine blades and NGVs.

In the next section the methods currently used for temperature monitoring of turbine blades and NGVs are discussed. Starting with numerical simulations used in the design stage, followed by offline systems used for validation, and finally online systems used in monitoring engines during normal operation. Summaries of the methods can be found in tables~\ref{tab:offline} and~\ref{tab:online}. The following section discusses an alternative temperature monitoring system that aims to address the limitations of the current methods, based on the use of ultrasonic guided waves. The most suitable methods of attaching sensors to the structure of a blade or vane are discussed, followed by an explanation of how Lamb waves can be utilised for this application.

\section{Current surface temperature monitoring strategies}\label{tempmonitoringmethods}

Numerical simulations are used in the design stage of blades and vanes to predict the efficiency of cooling techniques before parts are put into production. Experimental data from offline or online monitoring can then be compared to numerical simulations for validation. Offline systems record the peak temperatures of the blade for later analysis at room temperature while online measurements are carried out in real time. The majority of the offline systems are well established having been used for many years, with online systems being more difficult to implement and operate reliably as sensors need to be installed inside of the engine. The implementation of online systems is further complicated when applied to jet engines, as there are severe restrictions on space, weight, and power. The systems can be further categorised into either point measurement methods or mapping methods, where point-based systems only record the temperature at a single location, but can be installed in arrays to provide more spatial information. Mapping methods allow for continuous measurement across the surface of the blade, allowing for analysis of temperature gradients.

\begin{figure}[!htbp]
    \centering
    \includegraphics[width=.8\textwidth]{./figures/tempmethodsvenn.eps}
    \caption{Temperature monitoring systems for nozzle guide vanes}\label{fig:venn}
\end{figure}

\section{Numerical methods for temperature estimation}
Numerical simulations are used to find the cooling effectiveness of turbine blades and NGVs. Results of the simulations are validated against experimental results, such as those from temperature-sensitive paints~\cite{18-guan2019enhanced}. Finite element modelling (FEM) can be used to predict the thermal load on a blade and analyse the effectiveness of blade cooling methods. There are two methods of calculations, uncoupled or coupled. Uncoupled methods are less computational intensive as the heat transfer coefficients used in the calculations are only computed at key engine operating conditions. Coupled methods use computational fluid dynamic (CFD) calculations to iterate the heat transfer coefficients, which improves accuracy over uncoupled methods. Findeisen \textit{et al.}~\cite{Findeisen2017} compared two coupled methods, FEM1D and FEM2D. The FEM1D method uses simplified models of the cooling system making it suitable for initial designs, while the FEM2D method uses three dimensional CFD simulations that are more realistic but nearly a hundred times slower than FEM1D. This makes FEM2D models more suitable for comparison with experimental data. In order to validate complex thermal models over 80\% of the surface should be measured, which is not possible with point measurement systems such as thermocouples~\cite{Peral2019}. Thermal maps allow manufacturers to better understand the effects of air flow on specific areas of components, which can help to improve the life cycle of a part. Areas of increased thermal stress can be identified, which may require additional cooling or design adjustment, while also highlighting less critical areas that could be reduced in weight or cooling. 

\section{Offline Monitoring Methods}

Offline monitoring methods rely on an irreversible physical change to occur in order to accurately record the peak temperature of a surface. A summary of the available offline methods is shown in table~\ref{tab:offline}. Thermal paints and thermal history sensors are the most commonly used methods as they allow temperature to be mapped across the surface of the blade or vane while being less intrusive than other methods.

\subsection{Thermal paint or Temperature indicating paint}\label{thermalpaint} 
Specially formulated paints can be applied to the surface of a blade or vane that irreversibly change colour after exposure to high temperatures, recording the peak temperature at which they were exposed. Irreversible thermochromism takes place because of decomposition, i.e. changes to the crystal structure because of chemical reactions that produce new compounds. The change in colour is both a function of temperature and time, requiring careful control of exposure time for accurate analysis. Yang \textit{et al.}~\cite{Yang2015a} describe the development of temperature indicating paints including their formulation, application, and test methodology.

Paints can be categorised as single-change or multi-change, whereby single-change paints change colour once after exceeding a particular temperature threshold, and multi-change paints undergo multiple colour changes as temperature increases past a number of thresholds. Multi-change paints are more suitable for use on parts with large temperature gradients, such as turbine blades or NGVs. Examples of commercially available thermal paints produced by Thermographic Measurements Ltd, UK, can be seen in figures~\ref{fig:singlepaints} and~\ref{fig:multipaints}~\cite{Neely2006}.

\begin{figure}[!htbp]
    \centering
    \includegraphics[width=.8\textwidth]{./figures/Single change paints edit.png}
    \caption{Calibration colour maps for single-change paints: SC155, SC240, SC275, SC367, SC400, SC458, SC550, SC630~\cite{Neely2006}.}\label{fig:singlepaints}
\end{figure}

\begin{figure}[!htbp]
    \centering
    \includegraphics[width=.8\textwidth]{./figures/multi-change paints edit.png}
    \caption{Calibration colour maps for multi-change paints: MC135--2, MC165--2, MC395--3, MC490--10, MC520--7~\cite{Neely2006}.}\label{fig:multipaints}
\end{figure}

Results can be visually interpreted by an operator drawing isothermal lines, or analysed with automated image processing techniques. When analysis is carried out by an operator only a small number of temperature changes can be identified, where colour changes are obvious. In order to measure a greater range of temperature changes other blades can be monitored with different paints during the same test. Results from the different blades can then be combined to provide a more detailed map. When analysis is carried out using image processing techniques accuracy can be improved to identify colour changes representing $\pm$10\si{\degreeCelsius} changes, however inconsistent lighting conditions can cause significant sources of error. A large range of different paints can be used, covering the temperature range 150\si{\degreeCelsius}--1300\si{\degreeCelsius}. This range can be further extended to 1500\si{\degreeCelsius} with the use of fluorescent paint that is analysed under UV rather than visible light~\cite{Lempereur2008}, which reduces the chance of operator error and the effect of variable lighting conditions~\cite{Coleto2006}.

Engine test data is often used to validate numerical models. In order to produce quantitative results from the use of thermal paints, the engine test data must be compared with calibration data, without which the paints only provide qualitative spatial information. Laboratory-based tests can be carried out to replicate engine conditions while the temperature is monitored using other methods (e.g.\ thermocouples, IR cameras) for validation of the real engine data. Another option is to validate the real data using predictions based on the paint's chemistry~\cite{Neely2010}.

Colour changes are cross sensitive to environmental gases which must be accounted for when comparing results to calibration charts. Low paint durability means that testing time should be limited to between 3 and 5 minutes at peak temperatures for best results.~\cite{Bird1998}. Longer exposure time makes it difficult to determine if a colour change has occurred due to peak temperatures over a short period or lower temperatures over a longer period. Dismantling the engine for analysis can be time consuming and additional tests require the paint to be removed and reapplied before the engine is reconstructed. Many thermal paints contain cobalt, nickel, or lead compounds that are restricted for use by EU legislation (REACH~\cite{EuropeanParliament2006}) because of their toxicity.

\subsection{Thermal history sensors} 
Thermal history sensors are based on ceramic materials doped with luminescent transition or rare-earth ions. They are similar in principle to thermal paints in that exposure to high temperatures causes irreversible changes to their optical properties. The method was originally proposed by Feist in 2007~\cite{Feist2011}. Resolution and accuracy is improved in comparison to thermal paints as the optical change is continuous across the surface of the part and can be analysed with an emission detector, as with online thermographic phosphors. Parts do not have to be fully dismantled for analysis, and the phosphors do not contain restricted chemicals (EU REACH regulation~\cite{EuropeanParliament2006}). Biswas and Feist~\cite{KarmakarBiswas2014} discuss the theoretical background of thermal history sensors, their analysis methods, and carry out an experimental comparison against thermal paints and thermocouples. Results show good agreement between thermal paints and thermal history coating from 400\si{\degreeCelsius} to 900\si{\degreeCelsius}. Standard deviation of the thermal history coating measurement is typically below 5\si{\degreeCelsius}. Peral \textit{et al.}~\cite{Peral2019} has developed an uncertainty model for the use of thermal history sensors from 400\si{\degreeCelsius} to 750\si{\degreeCelsius}, indicating that the maximum estimated uncertainty was $\pm6.3$\si{\degreeCelsius} or $\pm 13$\si{\degreeCelsius} for 67\% or 95\% confidence levels respectively. This is believed to be well within the uncertainty of thermal models and the requirements for temperature measurements in harsh environments on gas turbines. 

Araguas Rodriguez \textit{et al.}~\cite{AraguasRodriguez2018a} have carried out temperature measurements between 350\si{\degreeCelsius} and 900\si{\degreeCelsius} on three types of NGVs (without cooling, internally cooled, externally cooled) and compared the results to an embedded thermocouple. The THP was made up of an oxide ceramic pigment mixed into a water-based binder doped with lanthanide ions. Measurements are carried out using a handheld probe, each taking approximately five seconds. Precision of the measurement was $\pm$5\si{\degreeCelsius}, and results were within the min-max range of the thermocouple results. An extended test (around 50 hours) using THPs was carried out by Pilgrim~\cite{Pilgrim2017}. No damage to the THP was observed, and results are in agreement with temperature sensitive paints and CFD models (10\si{\degreeCelsius} difference between them). This shows an improvement over temperature sensitive paints, which can only be used for short term ($\sim$5 minutes) engine tests. The processes behind the optical changes that are caused by temperature are explained by Rabhiou \textit{et al.}~\cite{Rabhiou2011}, followed by experimental results that show a Y$_{2}$SiO$_{5}$:Tb phosphor suitable for use up to 1000\si{\degreeCelsius}, with potential for further development to extend the upper temperature limit to 1400\si{\degreeCelsius}. A resolution comparison between temperature sensitive paints, pyrometry, and thermal history sensors is given by Amiel~\cite{Amiel2018} in Table 2. Pyrometry methods provide the highest resolution (0.3\si{\degreeCelsius}), followed by thermal history sensors (1--5\si{\degreeCelsius}), and finally temperature sensitive paints (10--100\si{\degreeCelsius}). Heyes~\cite{Heyes2012} carried out tests using a Y$_2$SiO$_5$:Tb phosphor suspended in IP 600 binder that could be used for temperature measurement from 400\si{\degreeCelsius} to 900\si{\degreeCelsius}. Results showed that the use of the binder caused some signal degradation, inhibiting the crystallisation process of the phosphor. This indicated the potential benefits of integrating the phosphors into thermal barrier coatings (TBCs), which negates the need for a binder. A preliminary investigation was carried out utilising air plasma spraying (APS) of YAG:YSZ:Dy which showed promising results. 

Although accuracy and resolution are improved in comparison to thermal paints, optical access to the parts is still required for analysis, which requires dismantling of the engine but not the parts themselves. Paints need to be removed and reapplied to carry out additional tests. The application of the paints requires sophisticated coating technologies which limits their uses. Careful phosphor selection is required as emission intensity decreases with increasing temperatures~\cite{Feist2001}.

\subsection{Other offline sensors} 
A number of other offline sensors have been developed for use on turbine blades or vanes that are described below. The use of these sensors is limited in comparison to temperature sensitive paints or thermal history sensors as they only provide point measurements while being more intrusive to blade operation.

Crystal temperature sensors have been developed to more accurately measure temperature gradients across blades in comparison with thermal paint methods. A crystal structure is installed on surface of the blade using thermo-cement and exposed to neutron radiation (“illumination”), causing the atomic lattice to expand. When exposed to heat the lattice relaxes, which can be analysed with an X-ray diffraction microscope. If exposure time is also measured the temperature can be deduced from comparison with a calibration diagram. Accuracy of the method is $\pm 10$\si{\degreeCelsius}, up to at least 1400\si{\degreeCelsius}~\cite{Annerfeldt2004}.

\begin{figure}
    \centering
    \includegraphics[width=.8\textwidth]{./figures/crystal temp sens.png}
    \caption{Crystal temperature sensors installed on a NGV~\cite{Annerfeldt2004}}\label{fig:crystal}
\end{figure}

Glass ceramic arrays rely on using materials that all undergo different allotropic phase transformations when exposed to temperature changes. Using a single material will result in the same phase change whether the material is exposed to a high temperature for a short time, or a low temperature for a long time, which can be accounted for when using a number of materials. As temperature and exposure time increases optical transmittance is reduced, which can be measured using a UV light spectrophotometer. A disadvantage of this method is that it is strongly influenced by moisture in the environment (as found in aircraft engine exhausts), which causes large changes in the crystallisation structure of the glass. It also suffers from an inability to account for overheat events unless they are of sufficient magnitude (\textgreater 50\si{\degreeCelsius})~\cite{Fair2008}. 

Metallurgical temperature sensors are a family of offline sensors that can be machined into almost any shape. They are best suited to applications where wire connections are not possible, especially those where the sensor is immersed in a corrosive medium. Their accuracy is poor in comparison to thermocouples (around 10\si{\degreeCelsius}) and they can only be used to give rough estimates of temperature. Stainless steel “Feroplugs” reduce in ferromagnetism with increasing temperature exposure, which can be measured using a ferritscope or magnetic suspectometer. The reduction in ferromagnetism fully diminishes above 600\si{\degreeCelsius}, the upper limit of temperature measurement. Sigmaplugs have been developed to extend the measurement range, introducing a new phase that can be measured up to 900\si{\degreeCelsius}. Templugs are made of steel alloys that reduce in hardness with increasing temperature. The change in hardness can be measured and compared with calibration charts to determine the temperature if exposure time is known. Accuracy is around 15\si{\degreeCelsius} up to 600\si{\degreeCelsius}, reducing to 25\si{\degreeCelsius} up to 650\si{\degreeCelsius}~\cite{Lo2012}. Madison \textit{et al.}~\cite{Madison2013} compared metallurgical temperature sensors with thermocouples when measuring piston temperatures in a running engine. Results were comparable however the metallurgical sensors were heavily influenced by areas with large thermal gradients.

\section{Online Monitoring Methods}

Online methods are used to measure temperature in real-time. This is particularly useful in comparison to offline methods as the monitoring can take place at start-up and cool down of the engine, rather than only providing a peak result. Another benefit over offline methods is the reduction in research and development time as the engine does not need to be dismantled between each test, which in turn reduces cost. This enables in-service data collection as opposed to short-term engine test data gathered from offline monitoring. Additional data helps to further validate numerical models when online systems are installed in test rigs, but they can also be used for condition monitoring applications on in-service engines. More comprehensive monitoring is beneficial in reducing maintenance schedules as these can be based on component health rather than a fixed number of operational hours. This has the benefits of reducing maintenance costs whilst minimising downtime. A reduction in the frequency of blade replacement is also beneficial to the environment. Table~\ref{tab:online} shows the currently available online monitoring methods. Thermographic phosphors are the most commonly used method as they are less intrusive than point-based thermocouple or RTD arrays, while being more reliable than pyrometry methods.

\subsection{Thermocouples} 
Conventional thermocouples are not suitable for use on turbine blades or NGVs, as their large size disrupts airflow, and reliably attaching them to the surface of a blade is difficult. Thin film thermocouples (TFTC) on the other hand can be directly fabricated on to the surface of a turbine blade or NGV, greatly improving long term durability in comparison to conventional thermocouples. COMSOL simulations carried out by Duan \textit{et al.}~\cite{Duan2020} show a ${\sim}$170\si{\degreeCelsius} difference between conventional thermocouples and their thin-film counterparts due to the distance between the thermocouple joint and the blade surface. The placement of a conventional thermocouple in its insulation filling adds an additional uncertainty of around $500\times500$ \si{\um}, which has a temperature difference from top to bottom of ${\sim}$300\si{\degreeCelsius}. The uncertainty associated with the use of conventional thermocouples shows the motivation for the development of thin film equivalents.

\begin{figure}
    \centering
    \includegraphics[width=.8\textwidth]{./figures/tftcNGV.png}
    \caption{TFTC array installed on a NGV~\cite{Duan2020}}\label{fig:tftcngv}
\end{figure}

The small footprint of TFTCs does not affect airflow, leading to more accurate measurements of fast temperature fluctuations. TFTCs have an extremely fast response time (\textless 1 \si{\us}), are low cost, and have very fine spatial resolution at a single point. S-type platinum plus platinum 10\% rhodium thermocouples have been shown to be stable up to 1100\si{\degreeCelsius}. However, oxidation of rhodium (which would occur in the gas path of a turbine) causes the thermoelectric potential to decrease, which in turn limits the upper working temperature of these types of thermocouple~\cite{Kreider1993}. The main challenges to overcome with TFTCs are: developing an electrical insulation film that can withstand the harsh conditions of the turbine (high temperatures, shock, and vibration), and applying the sensor to curved surfaces.

Yang \textit{et al.}~\cite{Yang2020} have shown that an S-type thermocouple can be fabricated using pulsed laser deposition (PLD) techniques that functions up to 700\si{\degreeCelsius}. To improve thermal stability at high temperatures composite ceramic oxides can be used, which have good anti-oxidation properties and have been shown to maintain high thermal outputs up to 1200\si{\degreeCelsius}~\cite{Liu2018}. Direct-Write Thermal Spray (DWTS) is a process by which powder, wire, or rods can be adhered to a complex substrate such as a blade or vane. A thermocouple can be embedded into the thermal barrier coating (TBC) on the surface of an NGV in this way and has been shown to perform well up to 450\si{\degreeCelsius}~\cite{Zhang2018b}. The addition of Al$_2$O$_3$ to YSZ(yttria-stabilized zirconia)-based TBCs improves their electrical insulation above 400\si{\degreeCelsius} which allows TFTCs to be used effectively~\cite{Gao2016}. A summary of THTC systems is given by Satish \textit{et al.}~\cite{Satish2017} along with the implementation of a K-type TFTC on an NGV deposited using an E-beam evaporation technique. Two methods of measuring temperature distribution on a TBC coating are given by Z. Ji \textit{et al.}~\cite{Ji2019} whereby Pt soldering dots are placed along a PtRH thin film. Measurements are possible up to 1200\si{\degreeCelsius} with this method.

Despite the improvement of thin film thermocouples over conventional thermocouples they are still an invasive monitoring method. Only point measurements are possible unless installed in dense arrays which can have an impact on blade operation, requiring complex wiring systems. The connection between thin-film and lead wires is the primary failure point when installed on turbine blades~\cite{Lei1997}. Oxidation has a significant impact on the operation of TFTCs at high temperatures, causing drift and delamination~\cite{Guk2016}, which makes the use of an insulation material vital.

\subsection{Pyrometry} 
Optical pyrometry is a non-contact method of measuring thermal radiation emitted from the surface of a material. There is no upper temperature limit as thermal radiation increases with temperature, but there is a lower limit of around 500\si{\degreeCelsius}~\cite{Kerr2004}. Measurement response is fast in comparison to conventional thermocouples as there is no thermal inertia to overcome. In order to implement a pyrometry system in a turbine optical access is required, which can be achieved by routing a probe through the wall of the turbine casing and connecting it to a detector via fibre optic cable, away from the high temperature environment of the turbine~\cite{Kerr2004}. The system can be considered a point measurement system as the probe is focused on an area 1--10 mm in diameter however the point can be scanned across the surface to produce temperature maps~\cite{Li2019}. A review of fibre optic thermometry methods is given by Yu \textit{et al.}~\cite{Yu2009a}, covering blackbody, infrared, and fluorescence optical thermometers. For accurate results the emittance of the blades must be measured prior to installation to later be used in correcting the output of the pyrometer, however the emittance is also affected by factors such as surface roughness, oxidation state, and chemical composition, which will affect the overall accuracy of the system. A number of solutions have been proposed to improve the accuracy of emittance measurement, namely dual and triple spectral pyrometry~\cite{Kerr2002, Li2018}. A review of multi-spectral pyrometry systems is given by Araújo \textit{et al.}~\cite{Araujo2017}. Accuracy is also affected by reflected radiation from other surfaces within the turbine, absorption of radiation in the gas path, interference from hot particles, and deposits forming on the surface of the probe lens. The effectiveness of pyrometry is negatively affected by the application of thermal barrier coatings (TBCs), which reduces emittance while increasing reflectivity of the blades. The thermal gradient of TBCs ($\sim$0.5\si{\degreeCelsius}/\si{\um}) means that careful wavelength selection is required to ensure that measurement is taking place at the required depth~\cite{Gentleman2006}. A number of authors have tested pyrometers on turbine engine test rigs~\cite{Becker1994,Frank2001,Wang2015}. Taniguchi \textit{et al.}~\cite{Taniguchi2006} have carried out temperature measurements using a pyrometer on a turbine blade rotating at 14,000 RPM. The system has a measurement range of 700--1150\si{\degreeCelsius} with an accuracy of $\pm$3\si{\degreeCelsius}, over a 2 mm diameter area. The measurement probe is only inserted into the hot section of the engine for a short time (10s) before being removed to avoid damage, which limits the long term use of this method. Infrared thermography has also been used on gas turbines for crack and defect detection rather than absolute temperature measurement~\cite{Hellberg}. Although pyrometry can provide high accuracy (in favourable conditions) and fast response times, the need for optical access and their large size restricts their uses to dedicated test rigs and large gas turbines. It is unlikely that a pyrometry system could be installed inside of a jet engine for SHM applications without considerable reductions in size and weight. During long term use the build up of deposits on the surface of the probe lens will reduce accuracy and require regular maintenance. In order to produce temperature maps the probe must be scanned across the surface which would require mechanical movement of the probe, a potential point of failure.

\begin{figure}
    \centering
    \includegraphics[width=.8\textwidth]{./figures/kerrpyro.eps}
    \caption{Cross sectional view of pyrometer installation~\cite{Kerr2004}}\label{fig:crosspyro}
\end{figure}

\begin{figure}
    \centering
    \includegraphics[width=.8\textwidth]{./figures/siemens IR edit-01.png}
    \caption{IR camera in use on a Siemens SGT-750 gas turbine~\cite{Hellberg}}\label{fig:IRsiemens}
\end{figure}

\subsection{Temperature sensitive paints} 
Temperature sensitive paints (TSPs) differ from thermal paints or temperature indicating paints in that they do not undergo permanent changes to their structure after exposure to high temperatures. The temperature dependence of luminescent molecules can be measured after application to the surface of a material using a polymer binder. The luminophore is excited by a light source, causing thermal quenching to take place. The intensity or decay lifetime of the luminescence can be measured using a camera, both of which are temperature dependent~\cite{Liu2005}. A selection of TSPs are described by Patel \textit{et al.}~\cite{Patel2016}, showing their useful temperature ranges. At low temperatures (\textless 200\si{\degreeCelsius}) metal-ligand complexes (MLCs) are used as luminophores, whereas at high temperature thermographic phosphors are used.

\subsection{Thermographic phosphors}\label{Thermographic phosphors}
Thermographic phosphors can be used in a very similar way to the metal-ligand complexes (MLCs) used in temperature sensitive paints however they are suitable for use up to much higher temperatures (${\sim}$1600\si{\degreeCelsius}). They can be excited and analysed in the same way as TSPs using laser excitation and fluorescence detectors. A number of authors have carried out reviews of thermographic phosphor systems, covering the principles and theory of fluorescence~\cite{Khalid2008a}, instrumentation~\cite{Allison1997}, implementation~\cite{Khalid2010}, and discussion of phosphor materials and error analysis~\cite{Brubach2013}. Phosphors can be applied to a blade or vane in a number of ways, the simplest of which is to mix the phosphor with a binder and apply it to the surface of the TBC, which provides a surface temperature measurement. A phosphor layer can also be applied beneath the TBC, which allows for monitoring of the temperature critical bonding layer. This is difficult to achieve as YSZ is mostly opaque to UV radiation, making excitation of the phosphor a challenge~\cite{AbouNada2016}. Phosphors can also be directly integrated into thermal barrier coatings (TBCs), acting as thermal insulation and as a sensor simultaneously. The first investigation into using phosphors embedded into thermal barrier coatings (TBCs), called ‘smart coating’ or ‘thermal barrier sensor coating’ was carried out by Feist~\cite{Feist2001}. Further development has shown that the addition of phosphors to produce sensor coatings does not affect the structure of TBCs and they can be applied using the same air plasma spray (APS) process as non-doped TBCs~\cite{Chen2005a}. The system has been applied to an NGV in a Rolls-Royce Viper 201 engine with optical access provided through a dedicated window~\cite{Feist2013}. A comparison between temperature measurements from a TBC doped sensor, pyrometer, and thermocouple shows a large temperature gradient between the surface of the TBC and the surface of the substrate, with the phosphor sensing taking place close to the bond coat interface~\cite{YanezGonzalez2015}.

Online temperature measurements from 513\si{\degreeCelsius} to 767\si{\degreeCelsius} have been carried out on an operating engine by Jenkins \textit{et al.}~\cite{Jenkins2020}. Measurements were carried out at rotational speeds up to 32,750 RPM. A laser was used for excitation of two phosphors applied to the TBC of the turbine blades, while a fibre optic probe was used for detection. Results were found to be within 25\si{\degreeCelsius} of of those estimated by the engine manufacturer. It is suggested that measurement range (from engine off to maximum engine power) can be improved by utilising Y$_{2}$O$_{3}$:Er phosphors~\cite{Eldridge2019}. Nau \textit{et al.}~\cite{Nau2019} have presented results from the use of five different phosphors that allow measurements from room temperature up to 1500\si{\degreeCelsius}. Comparison of the phosphors show that their individual temperature ranges are limited (400\si{\degreeCelsius}). Measurements were carried out on two model combustor systems, and at high pressure, to demonstrate the potential application in real turbines. Measurements are demonstrated with a high speed camera, allowing for temporal resolution up to 1 kHz.

\begin{figure}
    \centering
    \includegraphics[width=.8\textwidth]{./figures/yaneztempTBC.eps}
    \caption{A comparison of online monitoring systems (reproduced from Yañez Gonzalez \textit{et al.}~\cite{YanezGonzalez2015})}\label{fig:yaneztemp}
\end{figure}

The main limitation of using thermographic phosphors is the need for optical access, both for excitation and detection, which is normally achieved with fibre optic probes. Careful selection of phosphor material is required to cover the entire temperature range of interest and to ensure that emission intensity is high enough for accurate detection at elevated temperatures~\cite{Feist2001}. In the case of turbine blades the excitation and detection needs to be synchronised with blade rotation which requires extremely fast response times~\cite{Gentleman2006}, and is limited by the decay rate of the phosphor.  

\subsection{Other methods} 
Duan \textit{et al.}~\cite{Duan2018} have found that the electrically conductive nature of TBCs at elevated temperatures (above 600\si{\degreeCelsius}) can be utilised as a form of smart sensor, by measuring a change in resistance with temperature. Experimental data shows good repeatability up to 950\si{\degreeCelsius}, measurement error of less than 3\%, fast response time (${\sim}$1s), and stability comparable to conventional thermocouples and TFTCs.

\newgeometry{margin=3cm}  
\begin{landscape}

\begin{table*}[p]
        \resizebox{\linewidth}{!}{%
        \begin{tabular}{@{}llllll@{}}
        \toprule
        \textbf{Method} &   \textbf{\begin{tabular}[c]{@{}l@{}}Temperature \\ range (\si{\degreeCelsius})\end{tabular}} &
          \textbf{\begin{tabular}[c]{@{}l@{}}Temperature \\ resolution (\si{\degreeCelsius})\end{tabular}} &
          \textbf{\begin{tabular}[c]{@{}l@{}}Spatial \\ resolution\end{tabular}} &
            \textbf{Sensor type} &
            \textbf{Interrogation method} \\ \hline
        Crystal   temperature sensors & up to 1400 & $\pm$ 10     & Low                 & Surface mounted                  & Illumination + observation                \\
        Templugs                      & 650        & 15--25     & Low (single points) & Embedded in substrate            & Analysis of ferromagnetism/hardness/phase \\
        Glass   ceramic arrays        & 1100       & Not given & Low                 & Surface mounted                  & Observation only                          \\
        Thermal   paints (TIP)        & Up to 1500 & 10--100    & Moderate            & Surface layer                    & Observation only                          \\
        Thermal   history sensors     & 150--1400   & $\pm$ 5      & High                & Surface layer or embedded in TBC & Illumination + observation                \\ \bottomrule
        \end{tabular}%
        }
        \caption{\label{tab:offline} Summary of offline temperature monitoring methods}
\end{table*}

\begin{table*}[p]
    \resizebox{\linewidth}{!}{%
    \begin{tabular}{@{}llllll@{}}
    \toprule
    \textbf{Method} & \textbf{\begin{tabular}[c]{@{}l@{}}Temperature \\ range (\si{\degreeCelsius})\end{tabular}} & 
    \textbf{\begin{tabular}[c]{@{}l@{}}Temperature \\ resolution (\si{\degreeCelsius})\end{tabular}} & 
    \textbf{Spatial resolution} & \textbf{Sensor type} & \textbf{Interrogation method} \\ \midrule
    Thin film thermocouples (TFTCs)     & Up to 1200             & $\pm$ 2 & Low (can be installed in arrays) & Surface mounted                  & Electrical connection      \\
    Infrared Thermography/Pyrometry     & 500-\textgreater{}2000 & 0.3  & Low (can be scanned)             & Optical probe                    & Observation only           \\
    Temperature sensitive paints (MLCs) & \textless{}200         & 1--5  & High                             & Surface layer                    & Illumination + observation \\
    Thermographic phosphors             & Up to 1600             & 1--5  & High                             & Surface layer or embedded in TBC & Illumination + observation \\ \bottomrule
    \end{tabular}%
    }
    \caption{\label{tab:online} Summary of online temperature monitoring methods}
\end{table*}
\end{landscape}
\clearpage
\restoregeometry
\chapter{Emerging techniques for online monitoring}\label{ultrasonicmonitoring}

The ideal sensor for jet engine turbine blade and NGV temperature monitoring should be capable of mapping the temperature of a blade or vane in real time with minimum interference to operating conditions. The use of a sensor should not reduce the lifetime of the blade, cause damage to its structure, require large amounts of power, or require regular maintenance, ideally surviving for the lifetime of the engine. Response time should be fast in order to accurately record the changes in temperature during start-up, shut-down, and overshoot events, while resolution and accuracy should be comparable to traditional monitoring methods. 

Ultrasonic guided wave technology may be suitable for this application as it can provide real-time sensing, fast response times, and the ability to re-use the sensor indefinitely. There is the potential for transducers to be kept away from the harsh conditions of the turbine by transmitting a wave through the structure of a blade or vane and analysing the received signal. Measuring temperature in this way reduces the influence of the sensor on the operating condition of the blade. The small footprint of the sensors would not affect the operation of components or disrupt airflow. Advancements in sensor materials for use at high temperatures as well as the associated signal processing may allow for sensors to be installed in harsh environments. The following section considers the suitability of acoustic methods for the temperature monitoring of turbine blades and NGVs.

\section{Ultrasonic structural health monitoring}

Ultrasound is of particular interest for SHM applications as it allows for small sensors, high precision, fast data rates, and can be utilised at frequencies much higher than environmental noise~\cite{Mitra2016}. Traditional ultrasonic NDE utilises A-scans, a measurement of signal amplitude against time, to detect cracks, defects etc. This can be extended to B~\cite{Fatemi1980}, C~\cite{Ruzek2006,Imielinska2004}, or phased-array~\cite{Komura2001} scans to build an image of damage in an area by moving the transducers around and carrying out multiple measurements. This form of evaluation is not particularly suited to turbomachinery applications as the transducers would ideally be permanently installed on the structure. Guided wave SHM is of more interest as constructive interference with surfaces/boundaries allows waves to travel large distances with limited attenuation, which is already utilised for pipe~\cite{Zaghari2013} and rail inspection methods~\cite{Shi2019}. Ultrasound has been proposed as a method of defect detection NDE for aircraft~\cite{Wang2019}, installing transducers in large arrays to allow for guided wave tomography.

When using acoustic waves for SHM and NDE applications either a standalone sensor can be placed in an area of interest, or the structure of interest can directly be used as the sensing medium. In the context of high temperature turbomachinery it would be advantageous to avoid placing sensors into the gas path as they are difficult to interrogate and have the potential to affect operation of the engine components. The structure of turbine blades and NGVs can be used as the sensing medium assuming that an appropriate system of transducers can be implemented.

When considering using a structure as a sensing medium its geometry has an impact on wave propagation. The geometry of a typical turbine blade or NGV is that of a thin plate like structure through which ultrasonic guided waves will propagate. Rayleigh waves (otherwise known as ``Surface Acoustic Waves'' (SAWs)) can be excited at high frequencies (at wavelengths much smaller than material thickness) and are confined to the surface of a material, while Lamb waves (otherwise known as ``Guided Waves'') will propagate if excited at frequencies with wavelengths in the order of material thickness, as they interact with both the top and bottom boundaries of a material. Although Rayleigh waves are non-dispersive they produce large surface motions that are highly sensitive to any discontinuities or defects and are highly affected by surface coatings such as TBCs~\cite{Dransfeld1970}. Lamb waves will propagate through the multi-layered structure of substrate-bonding layer-TBC, where the through-thickness temperature gradient will affect wave propagation, rather than only the temperature at the surface of the TBC. The interaction between both boundaries of a material means that Lamb waves can be excited from the internal surface of an NGV, outside of the gas path. This reduces the influence of the sensor on turbine operation, and reduces the effect of temperature on sensor operation.

\subsection{Temperature compensation techniques} 
In many applications of Lamb wave SHM/NDE the effect of temperature on wave propagation is an undesirable factor that is compensated for using various signal processing techniques. These methods attempt to eliminate the influence of temperature in order to isolate the variable of interest i.e. defects. Temperature compensation or calibration can be carried out by using thermocouples~\cite{2-gajdacsi2014reconstruction}, but this is not ideal for the cases where the external use of a sensor is prohibited. Temperature compensation techniques, such as baseline signal stretch (BSS)~\cite{9R-12-croxford2007strategies, 1R-22-harley2012scale}, optimal baseline selection (OBS)~\cite{1R-24-konstantinidis2006temperature}, hybrid combination of BSS and OBS~\cite{1R-26-clarke2010guided, 1R-27-croxford2010efficient}, combination of OBS and adaptive filter algorithm~\cite{1R-28-wang2014adaptive}, Time of flight ($t_F$) calibration based on a linear relationship between $t_F$ and temperature~\cite{1R-li2018new}, Hilbert transform and the orthogonal matching pursuit algorithm~\cite{1R-20-liu2016baseline}, and temperature compensation for both velocity and phase changes~\cite{9-mariani2019compensation, 9R-16-herdovics2019compensation} have been proposed to reduce residuals between the baseline signal and the current signal. 

BSS method only targets the wave velocity change, neglecting the change in phase and amplitude of the signal. BSS method is restricted for small temperature variations. OBS method can accommodate larger temperature variations, however it requires many baseline signals that have sufficiently low post-subtraction noise levels at different temperatures. Therefore, a combination of BSS and OBS can achieve temperature compensation with a low number of baselines. A new method introduced by Mariani \textit{et al.}~\cite{9-mariani2019compensation} considers amplitude and phase changes as well as wave speed changes. It is shown that the residual between the baseline signal and current signals roughly halved when the two signals were acquired at temperatures 15\si{\degreeCelsius} apart~\cite{9-mariani2019compensation}. Since this method has not been used for measuring the temperature, the sensitivity of the technique has not been analysed. 

\section{Acoustic temperature sensing}

Bulk acoustic waves such as longitudinal and shear waves have been used for temperature sensing for a number of years. An overview of temperature sensing using acoustic waves is given by Lynnworth~\cite{Lynnworth1990}. Davis \textit{et al.}~\cite{Davis2008} experimented with an acoustic temperature sensor during variable frequency microwave curing of a polymer-coated silicon wafer. A sapphire buffer rod was used to separate the wafer from a Zinc Oxide (ZnO) transducer centred at approximately 600 MHz. Time of flight ($t_F$) was measured from the wafer/air interface reflection. Results were comparable to thermocouple measurements from 20\si{\degreeCelsius} to 300\si{\degreeCelsius}, $\pm$2\si{\degreeCelsius}. Takahashi \textit{et al.}~\cite{Takahashi2008} carried out 1D measurements of temperature distribution through a 30 mm thick steel plate in contact with molten aluminium (700\si{\degreeCelsius}). They noted that the system had a faster response time than the thermocouples used for validation of the method. Jia \& Skliar~\cite{Jia2016,Skliar2015} have demonstrated a method for ``UltraSound Measurements of Segmental Temperature Distribution'' (US-MSTD) in solids, utilising reflections along the signal path to estimate temperature distribution. Three different methods of parametrizing the segmental temperature distribution are discussed. The system has been tested in an oxy-fuel combustor at temperatures up to 1100 degrees, with results comparable to thermocouple measurements~\cite{Jia2019}. Jeffrey \textit{et al.}~\cite{Jeffrey2019} demonstrates 2D spatial ultrasonic temperature measurement through a container of wax using 8 0.5 MHz transducers (4 transmitters, 4 receivers), with results comparable to thermocouple measurements. Balasubramaniam \textit{et al.}~\cite{Balasubramaniam1999} developed a temperature measurement system in which an externally cooled buffer rod was used to separate an ultrasound transducer from the hot material under test (molten glass). Changes in time of flight ($t_F$) with temperature were measured from reflections from a notch placed close to the solid-liquid boundary. A calibration procedure carried out from 25\si{\degreeCelsius} to 1200\si{\degreeCelsius} was used to compensate for the thermal gradients of the delay line, and results were compared against a thermocouple  showing a greater than 2\% precision. It was noted that the change in $t_F$ due to temperature was greater than the change due to thermal expansion, leading to a non-linear relationship between temperature and time difference. Measurements were then carried out on molten glass with a resolution of 5\si{\degreeCelsius} (1 ns precision) using a 10 MHz transducer, however the author suggests resolution could be improved to 0.5\si{\degreeCelsius} with faster sampling. Ultrasonic oscillating temperature sensors (UOTSes) are another way to utilise ultrasound for temperature sensing whereby two transducers are setup to transmit through the material of interest in a feedback loop. A number of architectures are described by Hashmi \textit{et al.}~\cite{Hashmi2015, Hashmi2019}, with sensitivities up to 280 Hz/K~\cite{Alzebda2010}.

The uses of guided waves for temperature sensing purposes are limited. However, the fundamental antisymmetric Lamb wave mode, $A_0$, has been used for temperature monitoring of silicon wafers during rapid thermal processing~\cite{Lee1994a,Lee1996}. Quartz pins are used as waveguides, connecting to the wafer through Hertzian contact points. Time of flight ($t_F$) was measured at a rate of 20 Hz from 100\si{\degreeCelsius} to 1000\si{\degreeCelsius} with an accuracy of $\pm$5\si{\degreeCelsius} with this method. A laser excitation system has also been used to measure the temperature of silicon wafers during rapid thermal processing~\cite{Klimek1998}. A broadband excitation pulse is used, and two different methods of signal processing are compared. A cross correlation method between a room temperature baseline signal and those at elevated temperatures, plus a matrix comparison method whereby an unknown signal is compared to a database of signals taken over the whole temperature range of interest. The matrix method was shown to require less signal averaging than the cross-correlation method, with an accuracy in the order of 1\si{\degreeCelsius}.
The current implementations of acoustic temperature sensors demonstrate the potential of this method, but adapting these systems for use on turbine blades and NGVs will be challenging, starting with the method of effectively coupling to their structure. 

\subsection{Measurement devices for the generation and acquisition of guided waves}\label{wavegen}
The choice of transducer for the generation and acquisition of guided waves is limited by a number of factors including: elevated temperatures, space, and power consumption. When transmitting a wave through the structure of a material transducers can either be operated in pulse-echo mode or in a pitch-catch configuration. Pulse-echo mode operates with a single transducer acting as both transmitter and receiver, where the signal reflected from features such as defects or boundaries are analysed. Pitch-catch configurations operate with multiple transducers, some acting as transmitters and some as receivers. Response times are faster in pitch-catch configurations as waves travel directly between transducers rather than requiring a reflection to operate. Systems designed to map defects or damage (tomography) operate in pitch-catch configurations, usually with multiple pairs of transmitters/receivers to allow for higher spatial accuracy. For rotating turbine blades a transducer could be implemented in a pulse-echo configuration, if a transducer could not be housed at the rotating tip of the blade. Transducers could be operated in a pitch-catch configuration on NGVs as transducers could be placed on either side of the static vane. 

Piezoelectric based transducers are amongst the most common methods of generating and acquiring acoustic waves. They can be utilised in a number of different ways for either dedicated sensors or for transmitting a wave through a particular structure of interest. Many dedicated sensors are based on the use of interdigital transducers (IDTs) operated as either delay lines or resonators, and are often referred to as surface acoustic wave (SAW) devices. These sensors can be used for high temperature sensing and can be interrogated wirelessly~\cite{PereiraDaCunha2011,Dong2019}, however they would be difficult to implement on the structure of a turbine blade or NGV, while only providing a single point measurement. 

\begin{figure}[!htbp]
    \centering
    \includegraphics[width=.8\textwidth]{./figures/SAW delay line.png}
    \caption{A SAW delay line~\cite{Ida2014}}\label{fig:sawdelay}
\end{figure}

\begin{figure}[!htbp]
    \centering
    \includegraphics[width=.8\textwidth]{./figures/SAW resonator.png}
    \caption{A SAW resonator~\cite{Ida2014}}\label{fig:sawresonator}
\end{figure}

In order to transmit a wave through a structure using piezoelectrics there are a number of options, namely wafers and wedges. These sensors are reliant on an effective bond between transducer and substrate, which is difficult to a achieve at high temperatures.

Wedges (Figure~\ref{fig:wedge}) can be used to generate surface acoustic waves based on Snell’s law~\cite{Zhang2017a}, where a longitudinal wave produced by a piezoelectric transducer is transmitted through an angled wedge (typically made from a material with a slow longitudinal wave speed relative to the substrate, such as acrylic) into a substrate material where wave refraction takes place. Transmission of shear or surface acoustic waves are dependent on the material properties of the wedge and the substrate, and the wedge angle. A large benefit of this method is the ability to excite single Lamb wave modes in one direction~\cite{Khalili2016}. Unfortunately a liquid couplant is required to form an effective bond between wedge and substrate, which is unlikely to be suitable for permanent installation at high temperatures.

Piezoelectric Wafer Active Sensors (PWAS) are being used extensively for SHM applications and have been shown to withstand exposure to extreme environments~\cite{Mei2019}. They are non-resonant wide-band devices~\cite{Giurgiutiu2003a} however they can be used for generation of single Lamb wave modes with careful geometry selection~\cite{Ren2017}. PWAS are small, inexpensive, and minimally invasive~\cite{Giurgiutiu2003a}, making them potentially suitable for installation on turbine blades and NGVs if a suitable bonding method and piezoelectric material can be found.

\begin{figure}[!htbp]
    \centering
    \includegraphics[width=.8\textwidth]{./figures/wedge.eps}
    \caption{A typical wedge transducer}\label{fig:wedge}
\end{figure}

One promising solution to the bonding problem is to use a waveguide to separate the transducer from the extreme environment of the turbine. This method is especially suited to guided waves as they are known to travel large distances with little attenuation, however the strong reflections from the boundaries of the material can introduce dispersion, and material discontinuities should be considered~\cite{Parks2013}. One of the main challenges of this method is how to couple the waveguide to the structure of interest, as liquid couplants (as used for wedge coupling) are not suitable for use in high temperature environments, and simple welding is likely to introduce defects causing additional reflections and discontinuities to occur. It has been shown that clamping a waveguide to a structure can be effective even at relatively high temperatures (700\si{\degreeCelsius})~\cite{Cegla2011a}, although this method may not be suitable for installation on a turbine blade or NGV.
\\
\begin{figure}[!htbp]
    \centering
    \includegraphics[width=.8\textwidth]{./figures/hertzian.eps}
    \caption{An example Hertzian contact transducer}\label{fig:hertzian}
\end{figure}
\\
A system of Hertzian contact points (Figure~\ref{fig:hertzian}) may be more appropriate as only a small point would need to be in contact with the surface of interest. The contact size of the point determines the aperture size of the source, which can be considered as a point source up to very high frequencies, allowing for accurate measurement of absolute velocity~\cite{LeventDegertekin1997}. This method of coupling has been used to measure the mechanical properties of carbon fibre reinforced plastics (CFRP) using measured phase velocities of the $A_0$ and $S_0$ Lamb wave modes~\cite{Grimberg2010}. Application force of the rods can be controlled with springs. An example of a welded waveguide system installed on a turbine vane is given by Willsch~\cite{Willsch2004a}.

\begin{figure}[!htbp]
    \centering
    \includegraphics[width=.8\textwidth]{./figures/willschwaveguide.png}
    \caption{Willsch's welded waveguide~\cite{Willsch2004a}.}\label{fig:willsch}
\end{figure}

Other options for excitation/acquisition of acoustic waves include electromagnetic acoustic transducers (EMATs)~\cite{Khalili2018c} or lasers. Both of these methods could allow for contactless operation, however EMATs are quite inefficient, requiring large amounts of power to produce a signal of adequate amplitude. The use of lasers would require optical access to the surface of interest as well as the installation of a patch to protect the surface of the substrate from laser ablation~\cite{Kim2019a}.

Although the use of piezoelectric transducers is likely to be the most appropriate method of exciting acoustic waves for this application, temperature has an effect on their operation~\cite{4-jiang2014high, 12-fachberger2004properties}. The resonant frequency of a piezoelectric transducer reduces as temperature elevates~\cite{4-jiang2014high} and if the transducer operates at the resonance ringing may be seen at some temperatures~\cite{9-mariani2019compensation}. Filtering methods have been proposed to compensate for the transducer transfer function at different frequencies~\cite{ 9R-18-hurst2015real}. Temperature variations affect the bonding stiffness at the interface between the transducer and the structure. This results in frequency response changes which can affect both amplitude and the phase of the signal~\cite{9R-14-ha2010adhesive}. A common problem with all methods of excitation using piezoelectrics is their reduction in sensitivity with increasing temperature, which makes the choice of piezoelectric material vitally important.

\subsection{Piezoelectric selection for high temperature sensing} 
Piezoelectric materials for high temperature sensors have been compared by a number of authors~\cite{Turner1994,Tressler1998,Tittmann2013,Parks2013,Jiang2013b,Schulz2003,Zhang2011,Zhang2018}. Aluminium Nitride (AlN), Langasite (LGS), and Rare Earth Calcium Oxyborate Single Crystals (ReCOB), specifically Yttrium Calcium Oxyborate Single Crystals (YCOB), can be used for high temperature sensing with piezoelectrics~\cite{stevenson2015piezoelectric}.
YCOB has the highest operating temperature besides having a high resistivity at high temperatures. This signifies that YCOB will be able to operate at higher temperature as less heat is dissipated with a lower current. YCOB has relative small degradation in sensitivity with increasing temperatures of up to 1000\si{\degreeCelsius}, together with no significant phase change up to temperatures of 1500\si{\degreeCelsius}, which makes it ideal for high accuracy temperature measurements~\cite{zu2016high}. YCOB also has a linearly decreasing resistance with temperature, and a close to linearly decreasing resonance frequency with temperature, which can both be used for temperature measurement~\cite{zhang2008characterization}.

Aluminium Nitride (AlN) also exhibits a number of very promising properties necessary for high temperature sensing such as high electrical resistivity, temperature independence of electromechanical properties, and high thermal resistivity of the elastic, dielectric, and piezoelectric properties~\cite{Kim2015}. The use of an AlN sensor up to 800\si{\degreeCelsius} has been demonstrated for detection of laser generated Lamb waves in thin steel plates by Kim~\cite{Kim2018a}. Unfortunately high-quality AlN single crystals are difficult to grow, showing a wide range of resistivity that greatly affects their suitability as ultrasound transducers~\cite{Tittmann2013a}.

\section{Introducing Lamb waves for temperature sensing}\label{lambwavesensing}

The geometry of turbine blades and nozzle guide vanes allows Lamb waves to propagate through their structure at ultrasound frequencies. Lamb waves are a type of elastic wave present in thin plates when wavelength is in the order of thickness. They are guided by the upper and lower boundaries of a material allowing for continuous wave propagation (Figure~\ref{fig:lambwaves})~\cite{Rose2014}. The bulk acoustic waves discussed previously are non-dispersive, i.e their wave velocities are constant with frequency, whereas Lamb waves are dispersive and multi-modal which makes their analysis complex, especially when their are other factors such as changing temperatures are involved. The lowest order modes, the fundamental antisymmetric mode $A_0$, and the fundamental symmetric mode $S_0$ (Figure~\ref{fig:incophasedisp}), are the most commonly used modes as they are relatively non-dispersive and comparatively easy to generate in comparison to the higher order modes ($A_1$, $S_1$, etc.). Lower order Lamb waves are used extensively for NDE and SHM applications and an overview of their uses for damage identification is provided by Su~\cite{Su2009a}. Lamb waves have both phase and group velocities, the phase velocity relating to the local speed with which phase of the wave changes, and a group velocity
which describes the overall speed of energy transport through the propagating wave. Phase velocity is generally higher than the group velocity. Time of flight ($t_F$) measurements of Lamb waves give the group velocity, while special phase comparison techniques are needed to measure the phase velocity~\cite{Cheeke2000a}. 

\begin{figure}[!htbp]
    \centering
    \includegraphics[width=.8\textwidth]{./figures/lambwaves2.eps}
    \caption{Particle displacement of symmetric and antisymmetric Lamb wave modes in a plate.}\label{fig:lambwaves}
\end{figure}

\begin{figure}[!htbp]
    \centering
    \includegraphics[width=.8\textwidth]{./figures/inconelyoungsdensitypoisson.eps}
    \caption{Temperature dependent Young's modulus, density, and calculated Poisson's ratio for Inconel 718.}\label{fig:dpe}
\end{figure}

Lamb waves are affected by three main factors due to changes in temperature: thermal expansion, variations in Young's modulus, and transducer response (including bonding). The effect of temperature on density, Poisson's ratio, and Young's modulus, is shown in Figure~\ref{fig:dpe} for the superalloy Inconel 718~\cite{Abdullaev2019,SpecialMetals2007}. Nickel based superalloys such as these are commonly used for aerospace components that are exposed to high temperatures. A change in temperature has the greatest effect on Young's modulus in comparison to changes in thermal expansion (density and Poisson's ratio)~\cite{Abbas2020a}. These changes affect guided wave velocity~\cite{9R-10-weaver2000temperature}, an example of which is given in figure~\ref{fig:alutempshift} for the $S_0$ mode when excited in an aluminium plate using wedge transducers with a 5 cycle tone burst, where it can be seen that an increase in temperature from 20\si{\degreeCelsius} to 60\si{\degreeCelsius} reduces wave velocity and signal amplitude. A number of authors have investigated the temperature dependence of Lamb waves~\cite{Dodson2013,Marzani2012,Croxford2007a,Abbas2020a,Moll2019} however their studies are limited to relatively low temperatures. To better understand the uses of Lamb waves their frequency spectrum can be split into three areas:
\begin{itemize}
    \item Low frequency region – Contains the lowest two Lamb wave modes ($A_0$ \& $S_0$). It is possible to selectively excite either mode as they have very different wave velocities, leading to large differences in excitation angle when using a wedge transducer for example.
    \item Mid-frequency region – Contains many dispersive modes with similar wave speeds making it difficult to excite specific modes without exciting others, leading the production of complex waveforms that are difficult to analyse. 
    \item High frequency region – The modes become less dispersive and converge to similar wave speeds, leading to them travelling as a single packet. Group velocity can be measured for the packet. The $A_0$ \& $S_0$ modes begin to act like a Rayleigh wave as their velocities converge.
\end{itemize}

\begin{figure}[!htbp]
    \centering
    \includegraphics[width=.8\textwidth]{./figures/alutempshift2560.eps}
    \caption{The effect of temperature on the $S_0$ mode in an aluminium plate.}\label{fig:alutempshift}
\end{figure}

Figure~\ref{fig:incophasedisp} shows phase velocity curves for the symmetric and anti-symmetric Lamb wave modes generated by \href{https://www.dlr.de/zlp/en/desktopdefault.aspx/tabid-14332/24874_read-61142/#/gallery/33485}{The Dispersion Calculator}~\cite{Huber} for the superalloy Inconel 718. Dispersion curves such as these can be be generated based on material properties allowing areas of interest to be identified. 

\begin{figure}[!htbp]
    \centering
    \includegraphics[width=.8\textwidth]{./figures/phasedispersion.eps}
    \caption{Lamb wave phase velocity dispersion curves for Inconel 718.}\label{fig:incophasedisp}
\end{figure}

As an example of Lamb wave temperature dependence Figure~\ref{fig:groupshift} shows the shift in group velocity dispersion curves for the superalloy Inconel 718 from 21\si{\degreeCelsius} to 1093\si{\degreeCelsius}. An increase in temperature causes a reduction in wave velocity (shifting down) and a reduction in frequency (shifting left). The temperature dependence at a particular frequency can also be calculated, which is useful when determining a suitable excitation frequency. Careful selection of excitation frequency can allow for high temperature sensitivity depending on the mode excited and the dispersiveness of the region. As an example figure~\ref{fig:grouptempdependance} shows the temperature dependence of the $A_0$ \& $S_0$ modes at a frequency-thickness product of 1~MHz-mm. The change in wave velocity at this frequency-thickness product is non-linear because of two factors, a non-linear change in Young's modulus with increasing temperature, and the chosen frequency of 1~MHz falling into a more dispersive region as temperature increases, particularly for the $S_0$ mode. Average temperature sensitivity for the $A_0$ mode is \mbox{$-$0.898 m s$^{-1}$\si{\degreeCelsius}$^{-1}$} and \mbox{$-$1.868 m s$^{-1}$\si{\degreeCelsius}$^{-1}$} for the $S_0$ mode. It can be seen from Figure~\ref{fig:groupshift} that the $A_0$ mode has a relatively linear reduction in wave velocity with increasing temperature regardless of frequency, whereas the other modes have highly dispersive regions (steep slopes) that reduce in frequency (shift to the left) with increasing temperature. The high sensitivity to changes in temperature in these regions have great potential for temperature monitoring applications. Resolution is dependent on sampling frequency and the choice of time of flight ($t_F$) measurement method~\cite{Espinosa2018,Svilainis2013}. If a sampling rate of 2.5 GHz is used on the example given above over a distance of 10 mm (rough distance to first line of cooling holes) a theoretical velocity resolution of 0.4 m s$^{-1}$ for the $S_0$ mode and 0.2 m s$^{-1}$ for the $A_0$ mode can be achieved at 1093\si{\degreeCelsius}, which would allow for a temperature resolution of \textless{1\si{\degreeCelsius}} at 1~MHz. This would require measurements of $t_F$ at sub-wavelength resolution which, although challenging, can be achieved with cross-correlation methods~\cite{Jia2019a,Khyam2017}. Linear interpolation of cross-correlation methods can be used to increase resolution without increasing sampling rate~\cite{Costa-Junior2018}.

\begin{figure}[!htbp]
    \centering
    \includegraphics[width=.8\textwidth]{./figures/grouptemp.eps}
    \caption{$A_0$, $S_0$, $A_1$, and $S_1$ group velocity dispersion curve shift with temperature from 21\si{\degreeCelsius} to 1093\si{\degreeCelsius} for Inconel 718.}\label{fig:groupshift}
\end{figure}

\begin{figure}[!htbp]
    \centering
    \includegraphics[width=.8\textwidth]{./figures/grouptemp1mhztemp.eps}
    \caption{$A_0$ \& $S_0$ group velocity change with temperature from 21\si{\degreeCelsius} to 1093\si{\degreeCelsius} at 1~MHz for Inconel 718.}\label{fig:grouptempdependance}
\end{figure}

Although the non-dispersive nature of lower order modes ($A_0$ \& $S_0$) makes them relatively easy to analyse, it would be advantageous to operate at higher frequencies as phase shifts are easier to detect when wavelengths are shorter, which leads to improved sensing resolution and accuracy. As the wave speeds of the $A_0$ \& $S_0$ modes converge (around 10 MHz-mm for Inconel 718) they behave as a Rayleigh wave, which limits their use for this application as discussed previously. Higher order modes such as $A_1$ \& $S_1$ are more difficult to selectively excite as their phase/group velocities are similar, which leads to the formation of complex waveforms that are highly dispersive. However, as frequency-thickness product is increased further the excitability of higher order modes reduces dramatically which can allow less dispersive regions of lower order modes to be excited~\cite{Wilcox2004} (see high frequency region of figure~\ref{fig:incophasedisp}). Wilcox \textit{et al.}~\cite{Wilcox2003} have presented a method of reducing the effect of dispersion on a transmitted signal if prior knowledge of dispersion curves are known. This relaxes the need to excite only a single mode and simplifies the analysis process. In the case of turbine blades and NGVs, the propagation distance is relatively short so effect of dispersion is likely to be low, however a change in temperature will cause different regions of dispersion curves to be excited which will change the shape of a transmitted wave packet.

Jayaraman \textit{et al.}~\cite{Jayaraman2009a} have presented the existence of ``Higher Order Mode Cluster Guided Waves'' (HOMC-GW), a non-dispersive region found at high frequency-thickness products in which the various modes all have similar group velocities. This causes them to move as a single envelope, which can be treated like a single non-dispersive mode.  A number of aspects of HOMCs have been investigated including: their use for pipe inspections~\cite{Swaminathan2011, Chandrasekaran2010a, Balasubramaniam2008}, their interaction with weld pads~\cite{Verma2011}, and their interaction with notch-like defects in plates~\cite{SriHarshaReddy2017a}. Khalili \& Cawley~\cite{Khalili2016} carried out an investigation into exciting singular higher order modes which found that the HOMC described by Jayaraman was likely to be single mode dominating a cluster ($A_1$ around 20 MHz-mm). The use of this higher order mode cluster for temperature sensing is of particular interest for turbine blade applications and further investigation is required.

\section{Conclusion}

A review of the current methods of temperature sensing for turbine blades and NGVs in jet engines has been carried out. Offline systems such as thermal paints and thermal history sensors are well established, but provide limited data in comparison to online systems. When used for the validation of thermal models online systems can provide considerably more information, covering temperature changes through start-up to shut-down of an engine, as well as recording over shoot events. There are a number of online systems available including thin film thermocouples (TFTCs), thermographic phosphors, and pyrometers. Thermocouples need to be embedded into the structure of a component and require a wire connection which is a significant point of failure, while only providing point measurements unless installed in dense arrays. Pyrometers can provide temperature maps without installing sensors onto the surface of a component, but optical access is required, and environmental factors have a significant impact on their accuracy. Thermographic phosphors require optical access to components for both excitation and analysis, as well as direct application of a phosphor to the surface. This makes sensors difficult to implement for condition monitoring applications because of space constraints, especially in jet engines.

The ideal sensor for this application would operate outside of the gas path without interfering with operation of components, while still providing a high degree of accuracy with fast response times. Acoustic methods offer a potential advantage in their low power operation, small footprint, and temperature mapping potential. Further investigation is required to fully understand wave propagation through the complex geometries of turbine blades and NGVs. This includes coating materials and thicknesses, their temperature dependency, and their degradation (for example pitting and surface microcracking). Changes in surface characteristics can significantly alter the attenuation of guided waves and cracks/defects would cause additional reflections to occur. Residual stresses relating to high temperature gradients will also modify guided wave behaviour locally. Further investigation is required to determine the best method of attaching transducers to a blade or vane. Using waveguides to reduce the impact of temperature on the operation of transducers is likely to be the most appropriate option although there are a number of piezoelectric materials (most notably YCOB and AlN) that would be suitable for use at extremely high temperatures, allowing transducers to be mounted directly onto the blade housing.

The literature covering the temperature dependence of guided waves is limited to relatively low temperatures and generally only considers the lowest order fundamental Lamb wave modes ($A_0$ \& $S_0$). To achieve a temperature resolution that is comparable with traditional sensors the frequency of operation needs to be high in order to accurately detect a phase shift with changing temperature, which highlights the potential of higher order modes. Relative measurement of wave velocity is the most common method of temperature sensing using acoustic waves as in most cases (using non-dispersive waves) the change with temperature is linear. When utilising Lamb waves for this purpose careful excitation of single modes will reduce the complexity in signal analysis however this may not be possible with the limited range of transducers and mounting points available. Comparisons with baseline signals may be more appropriate which are already utilised in a number of different temperature compensation techniques. In order to map temperature distribution across a blade the complex series of reflections from cooling holes and boundaries need to be utilised. Reflections can cause mode conversion to take place that has the potential to add additional complexity to the system. This presents a signal processing challenge that has not been previously considered for this application. Further research is planned to evaluate the potential of a guided wave based temperature sensing system for turbine blades and NGVs. The temperature dependence of Lamb wave modes will be theoretically predicted and measured experimentally up to high temperatures. The results of this investigation can be used to determine the most suitable mode for temperature sensing. The interaction between guided waves and cooling holes will be investigated for the purposes of mapping the temperature across a blade or vane. Methods of coupling transducers to the structure of a blade will be considered based on the ability to excite the mode of interest and survive in the high temperature environment of a gas turbine.

\section{Other applications}

A Lamb wave based temperature monitoring system has potential for other applications. Batteries and battery packs are often arranged as thin cells, where the temperature of the cell gives an indication of cell health. Lamb waves would allow the monitoring of the cross-sectional temperature of a cell with the use of small piezoelectric sensors, giving an advantage over point-based systems such as thermocouples.
\chapter{Theoretical temperature sensitivity of Lamb waves}\label{theory}

From the outcome of the literature review, a temperature monitoring system for nozzle guide vanes based on the temperature dependent wave velocity of Lamb waves has been investigated. In this section the temperature dependence of various Lamb wave modes has been calculated theoretically by producing dispersion curves from material properties. Dispersion curves have been generated for Aluminium, to be validated experimentally in Section~\ref{experiments}.

\section{Generation of dispersion curves}

Theoretical dispersion curves calculated from the material properties of Aluminium have been produced using \href{https://www.dlr.de/zlp/en/desktopdefault.aspx/tabid-14332/24874_read-61142/#/gallery/33485}{The Dispersion Calculator}~\cite{Huber}. For isotropic media (Aluminium in this case) Rayleigh-Lamb equations are solved numerically to generate dispersion curves, based on the book by Rose~\cite{Rose2014}. The Rayleigh-Lamb frequency equations can be written as:

\begin{equation}
    \frac{\tan(qh)}{\tan(ph)}=-\frac{4k^2pq}{(q^2-k^2)^2}
\end{equation}

For symmetric modes, and:

\begin{equation}
    \frac{\tan(qh)}{\tan(ph)}=-\frac{q^2-k^2}{(4k^2pq)^2}
\end{equation}

For antisymmetric modes. Where $p$ is given by:

\begin{equation}
    p^2=(\frac{\omega}{c_L})-k^2
\end{equation}

And $q$ is given by:

\begin{equation}
    q^2=(\frac{\omega}{c_T})-k^2
\end{equation}

The wavenumber $k$ is numerically equal to $\omega/c_p$ , where $c_p$ is the phase velocity of the Lamb wave mode and $\omega$ is the circular frequency. The phase velocity is related to the wavelength by the simple relation $c_p = (\omega/2\pi)\lambda$. Group velocity is given by:

\begin{equation}
    c_g=c^2_p\left[c_p-(fd)\frac{dc_p}{d(fd)}\right]^{-1}
\end{equation}

Where $fd$ denotes frequency times thickness. Note that, when the derivative of $c_p$ with
respect to $fd$ becomes zero, $c_g = c_p$. Note also that, as the derivative of $c_p$
with respect to $fd$ approaches infinity (i.e., at cutoff), $c_g$ approaches zero.

\section{The effect of temperature on wave propagation}

A change in temperature will affect material properties, namely Young's modulus, Poisson's ratio, and Density. The change in density (thermal expansion) and Poisson’s ratio is negligible over the temperature of interest (20\si{\degreeCelsius}--100\si{\degreeCelsius}) in comparison to the change in Young’s modulus. Thermal expansion will cause a change in material dimensions which will also affect the wave speed, but the effect of this is small ($-$1.19 m s$^{-1}$ average change over 20--100\si{\degreeCelsius} for the $S_0$ mode). The change in Young’s modulus with temperature in Aluminium is close to linear, as represented by the Equation~\cite{Hopkins2012}: 
%
\begin{equation}
E\left( T \right) = \  - 4E7 \times T + 8E10\
\end{equation} 
%
Where $T$ is temperature in Kelvin, and $E$ is Young’s modulus in Pascals. 
The values of Young’s modulus produced by this equation have been used to generate the temperature dependant dispersion curves for aluminium shown in Figure~\ref{fig:grouptempshuftalu}. The angles of excitation required to excite the $S_0$, $A_1$, and $S_1$ modes are shown based on the frequency-thickness products of 1~MHz-mm (31\degree), 2.5~MHz-mm (24\degree), and 4~MHz-mm (28\degree) respectively. These frequency-thickness products correspond closely to group velocity maxima for each of the targeted modes, which is advantageous for separating the modes in the time domain~\cite{Alleyne1992}.

The group velocity of the $S_0$, $A_1$, and $S_1$ modes have been extracted from the curves at frequency-thickness products of 1~MHz-mm, 2.5~MHz-mm, and 4~MHz-mm respectively. The extracted group velocities are shown in Figure~\ref{fig:alugroupvel}. The temperature sensitivities of the modes are: 1.683--1.728 m s$^{-1}$\si{\degreeCelsius}$^{-1}$ for the $S_0$ mode, 0.928--1.018 m s$^{-1}$\si{\degreeCelsius}$^{-1}$ for the $A_1$ mode, and 1.444--1.548 m s$^{-1}$\si{\degreeCelsius}$^{-1}$ for the $S_1$ mode, over the temperature range 0--100\si{\degreeCelsius} (see Equation~\ref{eqn: eq k}). These values are frequency and mode dependent. The temperature sensitivities increase with temperature, as the modes reduces in frequency (shift left), and reduces in wave velocity (shift down). 

\begin{figure}[!htbp]
    \centering
    \includegraphics[width=.8\textwidth]{./figures/grouptempshiftalu.eps}
    \caption{$A_0$, $S_0$, $A_1$, and $S_1$ group velocity dispersion curve shift with temperature from 20\si{\degreeCelsius} to 100\si{\degreeCelsius} for Aluminium 1050 H14.}\label{fig:grouptempshuftalu}
\end{figure}

\begin{figure}[!htbp]
    \centering
    \includegraphics[width=.8\textwidth]{./figures/grouptemp1mhztempalu.eps}
    \caption{$S_0$, $A_1$, and $S_1$ group velocity change with temperature from 20\si{\degreeCelsius} to 100\si{\degreeCelsius} for Aluminium 1050 H14.}\label{fig:alugroupvel}
\end{figure}
\newpage
\section{Multi-modal Lamb waves}

The multi-modal nature of Lamb waves makes their analysis complex, however this can be simplified by targeting specific modes using careful selection of excitation frequency, to target a particular frequency-thickness product. This difficulty can be seen in Figure~\ref{fig:multimode} where a large number of different modes combine to create a multi-modal wave packet. Individual modes can be further isolated through the use of angled wedges, where the angle is chosen to target a particular mode. 

Below the cut-off frequency of the $A_1$ mode only the two fundamental modes ($A_0$ \& $S_0$) are present. Their phase/group velocities are sufficiently different from one another that the modes can be distinguished between in the time domain, assuming a long enough propagation distance. The use of wedge transducers can further isolate between the modes, as the wedge angle required to excite them are largely different. 

Above the cut-off frequency of the $A_1$ mode there are an increasing number of modes present. The relatively small difference in phase velocity between the higher order pairs ($A_1$ \& $S_1$, $A_2$ \& $S_2$, etc.) means that even with the use of wedge transducers it is not possible to selectively excite only one mode. The ability to differentiate between the modes is improved by targeting points of group velocity minima/maxima, where the modes are travelling with the largest differences in velocity, which helps to separate them in the time domain.

\begin{figure}[!htbp]
    \centering
    \includegraphics[width=\textwidth]{./figures/multimodeedit.eps}
    \caption{Simulated 100 mm wave propagation of 10--cycle 1~MHz sine pulse in a 4 mm thick aluminium plate.}\label{fig:multimode}
\end{figure}
\chapter{COMSOL simulations of Lamb wave propagation}\label{simulations}

The multiphysics simulation package COMSOL has been used to simulate potential temperature monitoring systems, investigating the effect of temperature on Lamb wave propagation.

\section{Variable angle wedge simulation}

A 2D model has been produced of the experimental test setup described in Section~\ref{experiments}. This allows for validation of the time of flight measurements, and can be used to separate the effect of temperature on the wedges from the substrate. The effect of temperature on the Lamb wave alone can therefore be analysed. 

The model consists of two variable angle wedges (PMMA), which are based on the geometry of the Olympus variable angle wedges used in the experimental investigation, placed on top of an aluminium plate. The thickness of the plate can be varied to target different Lamb wave modes at different frequency-thickness products. The initial thickness is set to 1 mm to target the $S_0$ mode at 1 MHz--mm. The transmitting wedge has a simplified piezoelectric transducer (PZT-5H) attached to it's rotating block, to which the excitation signal is applied. The geometry can be seen in Figure \ref{fig:COMSOLdiagram}. The received signal is measured at the receiver wedge's rotating block boundary. More realistic transducer configurations are not considered in this study, as the focus is on the effect of temperature on the propagating wave. A boundary area is set underneath the plate to act as the heat source, again mimicking the experimental setup. This is simplified to allow the temperature to be directly set, rather than simulating a hot plate.

\begin{table}[h]
    \centering
    \begin{tabulary}{\textwidth}{LLL}
        \hline
        \textbf{Property} & \textbf{PMMA} & \textbf{Aluminium}  \\
        \hline
        Heat capacity at constant pressure (J/(kg$\cdot$K)) & 1470 & 904 \\
        Density (kg/m$^3$) & 1190 & 2700\\
        Thermal conductivity (W/(m$\cdot$K)) & 0.18  & 237\\
        Young's modulus (Pa) & Equation~\ref{eqn:E PMMA} & Equation~\ref{eqn:E Alu} \\
        Poisson's ratio & 0.35 & 0.3375\\
        \hline
    \end{tabulary} 
    \caption{COMSOL material properties}\label{table:matprop}
\end{table}
%
%\begin{equation}\label{eqn:comsolexcitation}
   % 10 \times \sin{(2\pi f_{0}\text{t}) \times \left( \text{t} < \left( \frac{\text{np}}{f_{0}} \right) \right)}
%\end{equation}
% 
%Where ``np'' is the number of cycles in the pulse (10) and $f_0$ is the excitation frequency (1~MHz). The function is set to ``V'' as this will be applied to a voltage terminal of the simple piezoelectric transducer.
%
The change in Young's Modulus with temperature is included in the material properties for both the wedges (Equation~\ref{eqn:E PMMA})~\cite{Sahputra2018} and the aluminium (Equation~\ref{eqn:E Alu})~\cite{Hopkins2012} using piecewise functions.
%
\begin{equation}
    \label{eqn:E PMMA}
    E\left( T \right) = \ - 15250.0000000002\times T^2 + 1125000.00000015\times T + 5432499999.99998\    
\end{equation}
%
\begin{equation}
    \label{eqn:E Alu}
    E\left( T \right) = \  - 4E7 \times T + 8E10\   
\end{equation}
%
The change in Poisson's ratio and density is assumed to negligible and is not included in the simulation. Thermal expansion is also considered to have a negligible effect on the propagation distance and is excluded.

The modules Solid Mechanics, Electrostatics, and Heat Transfer in Solids are used in this simulation, along with a multiphysics node to couple Solid Mechanics with Electrostatics for the piezoelectric effect.

Both the wedges and the plate are set to isotropic linear elastic materials, with low reflecting boundaries applied to the wedges.

The simple piezoelectric transducer for the transmitting wedge is set up as follows: A zero charge node is used for the edges of the material, initial values are set to 0 V, a ``Charge Conservation, Piezoelectric'' node is set for the material, a ground boundary is selected for the wedge side of the material, and a terminal node is set for the opposite boundary. Within the terminal node the type is set to Voltage and the input is set to V0(t). The excitation signal is a 1 MHz 5--cycle Hamming windowed sine pulse.

For the Heat Transfer in Solids module all the domains are set to solid, and initial values are set to 20\si{\degreeCelsius}. The boundaries that are exposed to the air are selected in a Heat Flux node, where convective heat flux is selected. A user defined heat transfer coefficient of 15~W/(m$^2\cdot$K) is used. The external temperature is set to 20\si{\degreeCelsius}. The temperature of the boundary underneath the plate is adjusted as required.

The mesh size for each material is determined by excitation frequency. The excitation wavelength for each of the materials is calculated by dividing their longitudinal wave speed by $f_0$. A free triangular mesh is created for each of the materials, and the maximum element size for each of them is set to LocalWavelength/N. If higher frequency content is expected, the wavelength for each material should be based on the highest frequency expected rather than $f_0$. In order to accurately resolve a wave, at least 10--12 elements per local wavelength are required. This assumes linear discretization for all modules. Using 12 elements results in an average skewness rating (measure of element quality, 0--1) of 0.9345 over 154728 elements.

This study has two steps, firstly a stationary study to simulate the effect of temperature on the system until an equilibrium is reached, and secondly a time dependant study to simulate wave propagation that has it's initial conditions set by the stationary study. The settings for the initial study are adjusted to not solve for electrostatics/the piezoelectric effect. The time dependant study has it's ``Output times'' set to: range(0,dt,sim\textunderscore length) where ``dt'' is a global definition parameter equal to CFL/(N$\times f_0$). The CFL number is suggested by COMSOL to be less than 0.2, optimally 0.1. Under ``Values of Dependant Variables'' the settings are changed to user controlled, method is changed to Solution, and the study is set to the stationary study. The time step is manually set under Solver Configurations$>$Solution 1$>$Time dependant solver$>$Time stepping. Here the ``Steps taken by solver'' parameter is changed to ``Manual'' and the ``Time Step'' is set to: CFL/(N$\times f_0$). 

\begin{figure}[p]
    \centering
    \includegraphics[width=\textwidth]{./figures/comsoldiagram.png}
    \caption{COMSOL geometry diagram}\label{fig:COMSOLdiagram}
\end{figure}

\begin{figure}[p]
    \centering
    \includegraphics[width=\textwidth]{./figures/simmodes.png}
    \caption{Presence of the $A_0$ \& $S_0$ modes.}\label{fig:simmodes}
\end{figure}

\begin{figure}[p]
    \centering
    \includegraphics[width=\textwidth]{./figures/simpulse20cedit.eps}
    \caption{COMSOL simulation of $S_0$ mode propagation at room temperature.}\label{fig:COMSOLsimsignal}
\end{figure}

\section{Simulation results}

Exaggerated deformation of pressure in the plate as seen in Figure~\ref{fig:simmodes} makes the presence of the $A_0$ and $S_0$ modes clearly visible. The modes are separated in the time domain after a short distance ($\sim$~50 mm) due to the difference in group velocity. 

To visualise wave propagation and calculate time of flight the pressure at both transmitter and receiver wedge boundaries are exported, and the time of flight is measured using a cross-correlation method, to allow direct comparison with experimental results. An example of wave propagation at room temperature can be seen in Figure~\ref{fig:COMSOLsimsignal}. 

Calculated total Time of flight (through both the wedges and the plate) is longer than experimental measurements of the same setup. This causes calculated wave velocity to be lower than expected ($\sim$ -120 m s$^{-1}$). Time of flight in the wedge-to-wedge configuration is in line with experimental measurements, which eliminates the wedges as a source of error. The material properties of the aluminium plate are the same as those used in the theoretical study, which should (in theory) mean that the velocity in the simulated plate is the same as was extracted from dispersion curves. Frequency analysis of the transmitted wave shows that it is still centred at 1 MHz. 
\chapter{Experimental investigation of Lamb wave temperature sensitivity}\label{experiments}

The wave velocity temperature sensitivity of the $S_0$, $A_1$, and $S_1$ Lamb wave modes have been experimentally measured, to be validated against theoretical predictions in Chapter~\ref{theory}.

Two 1 MHz piezoelectric transducers attached to acrylic wedges (Olympus variable angle wedge) in a pitch-catch configuration have been coupled to a 1~mm thick aluminium plate with a liquid couplant (Figures~\ref{fig:testdiagram} \&~\ref{fig:testsetup}). A signal generator (Hewlett Packard 33120A) has been used to generate a 10-cycle tone burst at 1~MHz. Time of flight ($t_F$) is measured between transducers using a cross-correlation function in MATLAB (see Figure \ref{fig:crosscorr}), and wave velocity is calculated from the distance between transducers. Based on a sampling rate of 5$\times$10$^8$ the theoretical maximum temporal resolution is 2~ns. This is equal to a velocity resolution of $\pm$~0.35~m~s$^{-1}$.

\begin{table}[b]
    \centering
    \begin{tabulary}{\textwidth}{L}
        \hline
        \textbf{Measurement Hardware}     \\
        \hline
        2x Olympus ABWX-2001 Variable angle wedges \\
        2x Olympus A539S-SM 1 MHz transducers \\
        Olympus ultrasonic couplant B \\
        Hewlett Packard/Agilent 33120A Signal generator \\
        Picoscope 3406DMSO USB Oscilloscope \\
        Thermadata T-type temperature loggers \\
        VWR Hot plate \\
        \hline
    \end{tabulary} 
    \caption{Experimental measurement hardware.}\label{table:hardware}
\end{table}

\newpage

\begin{figure}[h]
    \centering
    \includegraphics[width=.8\textwidth]{./figures/testdiagramsimple.eps}
    \caption{Cross-sectional diagram of test setup.}\label{fig:testdiagram}
\end{figure}

Wave velocity is calculated using Equation~\ref{velocitycalc}/\ref{velocitycalcfull}. The propagation time through the wedges has been subtracted from the total $t_F$ to ensure that only the propagation time through the plate is measured.  
%
\begin{equation} \label{velocitycalc}
v = \frac{d}{t_F}
\end{equation} 
%
\begin{equation} \label{velocitycalcfull}
v = \left( \frac{d\;\text{between wedges} + d\;\text{wedge foot offset}}{\text{Total}\;t_F - \text{Wedge-to-wedge}\;t_F} \right)
\end{equation} 
%
\\
Where the $d$ wedge foot offset is 0.0425 m. This is the distance from the front edge of the wedge to where the wave enters the plate from the wedge. Measurements have been carried out at multiple temperatures (20\si{\degreeCelsius}--100\si{\degreeCelsius} in $\sim$10\si{\degreeCelsius} increments), controlled by a hot plate beneath the aluminium plate. 

The wedges allow for careful selection of excitation angle so that modes of interest can be targeted. The angle is determined based on Snell’s law:
%
\begin{equation} 
\text{Angle}\ \theta = \text{Sin}^{- 1} \left( \frac{\text{Longitudinal\ wedge\ velocity}}{\text{Lamb\ wave\ phase\ velocity}} \right)
\end{equation} 
%
Measurement of wave velocity depends on measurement of time of flight ($t_F$), which can be described by the Equation~\cite{Croxford2007a}:
%
\begin{equation}\label{tofcalc}
t_F = \frac{d}{c}
\end{equation} 
%
Where $d$ is the distance travelled at wave speed $c$, both of which are functions of temperature, $T$. The sensitivity of the time of flight to temperature can then be expressed as:
%
\begin{equation} 
\delta t_{F} = \frac{d}{c}\left( \alpha - \frac{k}{c} \right) \delta \text{T}
\end{equation} 
%
Where $\alpha$ is the coefficient of thermal expansion of the medium and $k$ is the rate of change of wave velocity with temperature:
%
\begin{equation} {\label{eqn: eq k}}
k = \frac{\delta \text{c}}{\delta \text{T}}
\end{equation} 
%
\begin{figure}[ht]
    \centering
    \includegraphics[width=.8\textwidth]{./figures/hjkhh7UmEx.png}
    \caption{Photograph of test setup.}\label{fig:testsetup}
\end{figure}

\newpage
\section{Test Method}

A hot plate is used to raise the temperature of the aluminium plate to the desired temperature. The temperature of the aluminium plate is monitored using thermocouples. The total $t_F$ is measured until it stabilises. The temperature of the entire system must be allowed to stabilise before taking the measurement to ensure that the temperature of the wedge is the same as the plate. Total $t_F$ is now measured for the set temperature. Multiple measurements are taken after adjusting wedge position. Wedges are removed from the surface and placed together to measure the wedge-to-wedge $t_F$. Multiple measurements are taken after adjusting wedge-to-wedge position. The $t_F$ measurement process is repeated after allowing the total $t_F$ to re-stabilise. Velocity is calculated using Equation~\ref{velocitycalcfull}. A mean average is calculated from the results of the repeated total $t_F$ measurements, and velocity is calculated for every wedge-to-wedge result. An average velocity is calculated along with standard deviation. 

The temperature gradient across the plate has been measured by placing four equally spaced thermocouples along the transmission path, from the centre of the plate to the furthest edge of the wedge transducer in 3 cm increments. The gradient is assumed to be the same on both sides of the plate. An average temperature has been calculated for the total transmission path at each hot plate temperature setting. 

\section{S0 mode (1 MHz-mm)}\label{S0 experiments}

\begin{figure}[ht]
    \centering
    \includegraphics[width=.8\textwidth]{./figures/waveformxcorr.eps}
    \caption{Time of flight ($t_F$) measurement using cross-correlation function at room temperature.}\label{fig:crosscorr}
\end{figure}
%
The wedge angle required for the $S_0$ mode is:
%
\begin{equation} 
31{^\circ} = \text{Sin}^{- 1} \left( \frac{2720}{5258} \right)
\end{equation} 
%
The $A_0$ mode cannot be excited using this method as it’s phase velocity at this frequency (2312 m s$^{-1}$) is slower than the longitudinal velocity of the wedge. If the $A_0$ mode is present in the signal it will not affect measurement of the $S_0$ mode as it’s group velocity is significantly different than that of the $S_0$ mode, which will cause a distinct second wave packet. 

\subsection{S0 mode results}

\begin{figure}[ht]
    \centering
    \includegraphics[width=.8\textwidth]{./figures/aluplatemeasured.eps}
    \caption{Group velocity change with temperature for the $S_0$ mode in Aluminium 1050 H14.}\label{fig:result}
\end{figure}

Figure~\ref{fig:result} shows experimentally measured wave velocity of the $S_0$ mode plotted against theoretical wave velocity extracted from dispersion curves. Error bars show the standard deviation from the mean. After accounting for the temperature gradient across the transmission path by calculating a temperature average the change in velocity is comparable with predicted velocity extracted from dispersion curves, within 4.89 $\pm$ 2.27 m s$^{-1}$ on average. The temperature sensitivity of the system is 1.26--1.78 m s$^{-1}$\si{\degreeCelsius}$^{-1}$ over the range 24\si{\degreeCelsius}--94\si{\degreeCelsius}. The sensitivity is extracted from a second-order polynomial fit of the data (r$^2$ = 0.9992). The slope away from predicted results (increasing with temperature) can be attributed to the increasing temperature gradient, both in the plate and in the wedges. The gradient is shown to be almost linear (r$^2$=0.9967) across the measurement distance. Increasing temperature is also likely to have an effect on the operation of the piezoelectric transducer (amplitude and centre frequency), however this effect is negligible over the tested temperature range. The wedge angle required to excite the $S_0$ mode will also vary with temperature, however the change is only around 1\si{\degree} between 20\si{\degreeCelsius} and 100\si{\degreeCelsius}.
\newpage
\section{A1 mode (2.5 MHz-mm)}

The wedge angle required for the $A_1$ mode is:
%
\begin{equation} 
24{^\circ} = \text{Sin}^{- 1} \left( \frac{2720}{6654} \right)
\end{equation} 
%

\subsection{A1 mode results}

Figure~\ref{fig:A1result} shows experimentally measured wave velocity of the $A_1$ mode plotted against theoretical wave velocity extracted from dispersion curves. Error bars show the standard deviation from the mean. After accounting for the temperature gradient across the transmission path by calculating a temperature average the change in velocity is comparable with predicted velocity extracted from dispersion curves, within 2.43 $\pm$ 1.97 m s$^{-1}$ on average. The temperature sensitivity of the system is 1.09--1.17 m s$^{-1}$\si{\degreeCelsius}$^{-1}$ over the range 26\si{\degreeCelsius}--97\si{\degreeCelsius}. The sensitivity is extracted from a second-order polynomial fit of the data (r$^2$ = 0.9990).

\begin{figure}[h!]
    \centering
    \includegraphics[width=.8\textwidth]{./figures/a1moderesult.eps}
    \caption{Group velocity change with temperature for the $A_1$ mode in Aluminium 1050 H14.}\label{fig:A1result}
\end{figure}
\newpage
\section{S1 mode (4 MHz-mm)}

This region of frequency-thickness product is multi-modal, with both the $A_1$ and $S_1$ modes present. Similarities in phase velocity leads to similar excitation angles, which causes both modes to be excited. Using a cross-correlation method for measuring time-of-flight is no longer appropriate, as the received signal differs substantially from the input signal. An envelope peak method is employed instead, whereby a peak envelope for both the excitation signal and received signal is generated by spline interpolation over local maxima, separated by 500 samples. This produces a smooth envelope with clearly defined central peaks, as seen in Figure~\ref{fig:S1timedomain}. A peak finding algorithm is used to detect the envelope peaks, as denoted by the dashed lines. The $S_1$ mode ($\sim$~4550 m s$^{-1}$) arrives at the receiver before the $A_1$ mode ($\sim$~2550 m s$^{-1}$) as it has a considerably higher group velocity. At this propagation distance the two modes are clearly separated in the time domain, with the $A_1$ mode showing considerably more dispersion.  

The wedge angle required for the $S_1$ mode is:
%
\begin{equation} 
28{^\circ} = \text{Sin}^{- 1} \left( \frac{2720}{5874} \right)
\end{equation} 
%
\subsection{S1 mode results}

Figure~\ref{fig:S1result} shows experimentally measured wave velocity of the $S_1$ mode plotted against theoretical wave velocity extracted from dispersion curves. Error bars show the standard deviation from the mean. After accounting for the temperature gradient across the transmission path by calculating a temperature average the change in velocity is comparable with predicted velocity extracted from dispersion curves, within 4.44 $\pm$ 7.15 m s$^{-1}$ on average. The temperature sensitivity of the system is 1.80 m s$^{-1}$\si{\degreeCelsius}$^{-1}$ over the range 25\si{\degreeCelsius}--103\si{\degreeCelsius}. The sensitivity is extracted from a linear fit of the data (r$^2$ = 0.9777).

\begin{figure}[h!]
    \centering
    \includegraphics[width=.8\textwidth]{./figures/S1modetimedomain.eps}
    \caption{Time of flight ($t_F$) measurement of $S_1$ mode using envelope peak method at room temperature.}\label{fig:S1timedomain}
\end{figure}

\begin{figure}[h!]
    \centering
    \includegraphics[width=.8\textwidth]{./figures/s1moderesult.eps}
    \caption{Group velocity change with temperature for the $S_1$ mode in Aluminium 1050 H14.}\label{fig:S1result}
\end{figure}

\newpage
\section{Experimental sensitivity analysis}

There are a number of experimental error sources to consider. The physical distance between wedges is controlled using 3D printed spacers that keep the wedges aligned at set distances. The movement of the wedges on the surface of the plate increases with temperature as the viscosity of the couplant decreases. Variations in placement cause the calculated velocity to vary by around $\pm$ 5 m s$^{-1}$ across multiple (30) wedge re-alignments. The measurement of wedge-to-wedge time to be subtracted from the total $t_F$ is temperature dependant and relies on accurate alignment of the wedge feet, as well as a good connection between them (signal amplitude is highly dependant on couplant). Variation in alignment causes around a $\pm$ 10 m s$^{-1}$ velocity change. The measurement of the distance from the front edge of the wedge to the point at which the wave enters the plate from the wedge (``wedge foot offset'' in Equation~\ref{velocitycalcfull}) has a large effect on the calculated wave velocity. The exact offset distance is unknown and is assumed to be the point at which the centre line of the transducer aligns with the plate surface. Varying this value (42.5 mm) raises or lowers the velocity of all results considerably ($\pm$ 1 mm = $\pm$ 35 m s$^{-1}$). The hot plate does not heat the test plate evenly, especially at distances greater than 10 cm between wedges where they overhang the edges of the hot plate. The gradient (the difference in temperature between the centre of the plate and the location of the wedges) increases with temperature. The measured velocity is monitoring the average temperature of the transmission path. The gradient has been measured by placing a number (4) of thermocouples along the transmission path, from the centre of the plate (maximum temperature) to the point at which a wave is transmitted between a wedge foot and the plate. The calculation of aluminium dispersion curves at different temperatures is based on a change in Young's modulus. This is predicted from Hopkin's formula \cite{Hopkins2012} that may not give the correct values for Aluminium 1050 H14, but Aluminium in general.

The largest source of error is the measurement of wedge-to-wedge $t_F$, as a small error in alignment causes a large change ($\pm$ 10 m s$^{-1}$) in the wave velocity calculation. This is accounted for through the averaging of multiple (30) measurements, the standard deviation for this range is shown using error bars on Figure~\ref{fig:result}. Selection of wedge foot offset distance dramatically shifts the calculated velocity. This value is difficult to measure to an accuracy of \textless~1 mm. 

\section{Conclusion}

The theoretical effect of temperature on various Lamb wave modes in aluminium plates has been investigated by generating dispersion curves based on varying material properties (Figure~\ref{fig:alutempshift}). This can be repeated in the future for other materials at higher temperatures (e.g.\ Inconel 718 up to 1100\si{\degreeCelsius} in Figure~\ref{fig:groupshift}). The temperature sensitivity of the $S_0$ mode at 1~MHz has been extracted from these curves (Figure~\ref{fig:alugroupvel}) to be validated experimentally.

An experimental investigation has been carried out in order to validate theoretical predictions. Wedge transducers in a pitch-catch configuration have been used to excite the $S_0$ mode in a 1mm thick aluminium plate. The time of flight between transducers has been measured using a cross-correlation method and wave velocity calculated based on the distance between transducers. This confirms that the $S_0$ mode has been excited. The change in $S_0$ wave velocity due to temperature is in line with theoretical predictions over the range 20\si{\degreeCelsius}--100\si{\degreeCelsius} as shown in Figure~\ref{fig:result}. 

It is clear that wedge transducers are not the optimum method of transmitting/receiving a wave through a nozzle guide vane at high temperatures. They cannot be permanently mounted to the structure due to the need for a liquid couplant, and their relatively large footprint would make finding a suitable mounting location a challenge. Their operation at high temperatures is limited by the wedge material, which in the case of acrylic melts at around 160\si{\degreeCelsius}. The wedge material needs to have a longitudinal wave velocity less than that of the targeted Lamb wave phase velocity, which limits the choice of material severely, mostly to plastics with low melting points. The great benefit of wedge transducers is the ability to selectively target Lamb wave modes, which reduces the complexity of data analysis compared with exciting multiple modes simultaneously. This is difficult to achieve using other transducer configurations but it may instead be possible to excite a higher order region that travels as a single wave packet. The use of PWAS transducers could allow for operation at high temperatures (assuming suitable choice of piezoelectric material) and would be relatively easy to mount to an NGV structure, having a small footprint, although the high temperatures are likely to make bonding difficult. Another option is to couple into the structure using waveguides, distancing the transducers from the high temperature environment. Future research will investigate the temperature sensitivity of higher order modes (such as $A_1$ and $S_1$), as operating at higher frequencies can improve resolution (allowing for the detection of smaller phase shifts) and response rates. The ability to monitor wave velocity variations in multi-modal wave packets will also be considered when investigating transducer configurations suitable for higher temperature operation.
\chapter{Research summary \& future plans}\label{plan}

\section{Research summary}

Prior to the 18 month confirmation review the following research has been undertaken:

\begin{itemize}
    \item Evaluation of the currently used methods for temperature monitoring of NGVs/turbine blades, with a focus on online methods. The outcome of this study can provide the requirements for the development of a new system. This is addressed in Section~\ref{tempmonitoringmethods}.
    \item An investigation into whether ultrasonic methods are a suitable alternative to the currently available monitoring systems, based on a number of factors such as: response time, accuracy, reliability, power consumption, and possible sensor installation methods. This is addressed in Section~\ref{ultrasonicmonitoring}. This study contributes to answering research question~\ref{itm:1}.
    \item Identifying a method of exciting Lamb waves in plate-like structures. This is addressed in Section~\ref{wavegen}. This study contributes to answering research question~\ref{itm:2}.
    \item Understanding the effect of temperature on Lamb wave propagation. This is addressed in Section~\ref{lambwavesensing}. This study contributes to answering research question~\ref{itm:1}.
    \item Verification of the theoretical temperature sensitivities of Lamb wave modes with experimental data. Measurements up to 100\si{\degreeCelsius} in an aluminium plate. This is addressed in Section~\ref{theory}. This study contributes to answering research question~\ref{itm:2}.
    \item COMSOL modelling of the experimental test system. The model will be used in future studies. This is addressed in Section~\ref{simulations}. This study contributes to answering research questions~\ref{itm:3},~\ref{itm:4},~\ref{itm:5}, and~\ref{itm:6}.
\end{itemize} 

%Considering the outcome of the experimental study it has been shown that temperature can be monitored using ultrasound techniques. The next stage of the study is to investigate the effect of cooling holes on the propagation of waves, and to explore the possibility of monitoring temperature in multiple locations by analysing the reflections from a number of holes.

%In order to adapt the system to nozzle guide vanes a different transducer configuration is required that can be permanently coupled to the structure, and can operate at considerably higher temperatures than the current system. As shown previously the most suitable candidate for achieving this are waveguides that can be used to distance the transducers from the high temperature environment. A transducer configuration that utilises waveguides will be modelled using COMSOL, and the suitability of the system for use on NGVs will be evaluated. Based on the outcome of this study, and time permitting, a waveguide system will be tested experimentally, in order to validate the model. %

\newpage
\section{Future plans}

\begin{itemize}
    \item The effect of plate holes on Lamb wave mode propagation will be investigated through simulation and experimentation. This will mimic the cooling hole structure found on NGVs. The effect on amplitude and mode conversion will be investigated for a number of modes, to determine the most suitable mode for temperature monitoring. This study will partially answer research questions~\ref{itm:3} and~\ref{itm:4}.
    \item The ability to monitor temperature at a number of locations through the analysis of acoustic reflections will be explored. This will follow on from the previous study, focusing on the use of the mode previously determined to be most suitable. Both pulse-echo and pitch-catch transducer configurations will be investigated. This study will answer research question~\ref{itm:5}, and contribute to answering research question~\ref{itm:6}.
    \item The effect of thermal barrier coatings (TBCs) on wave propagation will be investigated using COMSOL models. This study will partially answer research question~\ref{itm:3}. 
    \item To operate at higher temperatures a new transducer configuration will be required. The most suitable option is to distance the transducers away from the high temperatures using waveguides, coupling into the test structure using a system of Hertzian contact points. COMSOL simulations will be used to investigate the potential of this method. This study will partially answer research question~\ref{itm:6}. 
    \item Depending on the outcome of the previous COMSOL study and time constraints, a waveguide system will be tested experimentally. This study will partially answer research question~\ref{itm:6}. 
\end{itemize}

\section{Gantt chart}

\includepdf[landscape=true,fitpaper=true]{./figures/gantt.pdf}

%\printbibliography
\bibliographystyle{ieeetr}
\bibliography{library.bib, References.bib}
\end{document}