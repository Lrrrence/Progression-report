\chapter{Nozzle guide vane (NGV) temperature monitoring}\label{literev}
This literature review considers surface temperature monitoring methods for turbomachinery applications, focusing on jet engine turbine blades and nozzle guide vanes (NGVs).

Carbon dioxide emission is the largest contributor from aviation to global warming, comprising 70\% of aircraft engine emissions~\cite{administrator2015aviation}. A small improvement in the design of a jet engine will result in a sizable reduction in pollution, as well as in fuel and engine costs~\cite{zhang2019number,Zaghari2020}. The results of temperature monitoring can inform the research and development process to improve the operation and efficiency of components~\cite{Zaghari2017}. More accurate monitoring in terms of absolute temperature, spatial resolution, and time history allows for further development, maximising the effectiveness of components. Current and past methods focus on testing parts in controlled environments before installation in an engine, with the majority of methods requiring the blade or vane to be removed from the engine for analysis after a test has taken place. This can be improved with online methods that allow for monitoring during normal engine operation. Online methods provide considerably more data for analysis during start-up and cool down of the engine rather than only providing a peak result. The currently available online methods can only provide point measurements (in the form of thermocouples) or require optical access (pyrometry \& thermographic phosphors) which is difficult to achieve in the cramped conditions of a turbine. These online systems have been utilised in test rigs and gas turbines but have not been adapted for use on in-service jet engines because of space, weight, and power constraints. An improvement to these sensors would greatly benefit gas turbine manufacturers, allowing for further development in turbine blade and NGV design by better understanding their limits. If the uncertainty of temperature measurements can be reduced then engine efficiency can be improved by operating components closer to their ideal conditions~\cite{Kerr2002}.

An increase in turbine inlet temperature elevates the risk of blade failure from blade creep or oxidation, the extent to which is determined by the exposure time at raised temperatures~\cite{Becker1994}. Blades and vanes require constant cooling with film cooling techniques embedded into their structure to avoid corrosion and melting~\cite{Zhou2016}, with gas temperatures up to 1000\si{\degreeCelsius} using uncooled blades and around 1800\si{\degreeCelsius} with cooled blades~\cite{Dixon2013}. The air used for film cooling is drawn from the main gas path, reducing efficiency while increasing NO\textsubscript{X} emissions~\cite{Heyes2004}. As emission limits become stricter the use of film cooling may have to be reduced, emphasising the importance of temperature monitoring to allow components to be operated closer to their thermal limits. 
Components such as turbine blades and NGVs are coated with ceramic based thermal barrier coatings (TBCs) to allow gas stream temperatures to be further increased. These coatings are normally multi-layered, utilising yttrium-stabilised-zirconia (YSZ; Y$_2$O$_3$–ZrO$_2$) as the insulating material, a thermally grown oxide layer (TGO), and a bonding layer to attach to the substrate~\cite{Thakare2020}. The bonding layer is considered to be the most likely point of failure from exposure to excessively high temperatures~\cite{Feist2001}, monitoring of this area is therefore vital for maintaining blade health. 
Reviews of condition monitoring/non-destructive evaluation (NDE)/structural health monitoring (SHM) for all aspects of gas turbines are provided by Abdelrhman~\cite{Abdelrhman2014} and Mevissen~\cite{Mevissen2019}, however there are a limited number of studies that focus on developments in online surface temperature monitoring for jet engines, which is covered in the first half of this literature review.

\begin{figure}[!htbp]
    \centering
    \includegraphics[width=.8\textwidth]{./figures/Turbinecrosssection.eps}
    \caption{Cross section of a typical turbine blade with TBC applied~\cite{AbouNada2016}}\label{fig:turbineblade}
\end{figure}

Acoustic methods are a potential alternative that have already been utilised for a number of other SHM and NDE applications. Ultrasonic guided waves in particular offer many benefits including low costs, high accuracy, and fast response times if utilised correctly. Acoustic waves transmitted through the structure of a blade or vane would not interfere with their operation, and temperature mapping could be achieved by utilising the acoustic reflections from the large number of cooling holes found across the structure. The difficulties of this method are in managing the impact of high temperatures on electronic components~\cite{Zaghari2017a} and sensor operation, which requires appropriate material selection and signal processing techniques. There are a number of different options for the generation and acquisition of waves, methods of coupling to the structure, and temperature derivation, all of which need to be investigated further to determine the most suitable method. The second half of this literature review discusses the potential of acoustic methods and how they can be implemented for use on turbine blades and NGVs.

In the next section the methods currently used for temperature monitoring of turbine blades and NGVs are discussed. Starting with numerical simulations used in the design stage, followed by offline systems used for validation, and finally online systems used in monitoring engines during normal operation. Summaries of the methods can be found in tables~\ref{tab:offline} and~\ref{tab:online}. The following section discusses an alternative temperature monitoring system that aims to address the limitations of the current methods, based on the use of ultrasonic guided waves. The most suitable methods of attaching sensors to the structure of a blade or vane are discussed, followed by an explanation of how Lamb waves can be utilised for this application.

\section{Current surface temperature monitoring strategies}\label{tempmonitoringmethods}

Numerical simulations are used in the design stage of blades and vanes to predict the efficiency of cooling techniques before parts are put into production. Experimental data from offline or online monitoring can then be compared to numerical simulations for validation. Offline systems record the peak temperatures of the blade for later analysis at room temperature while online measurements are carried out in real time. The majority of the offline systems are well established having been used for many years, with online systems being more difficult to implement and operate reliably as sensors need to be installed inside of the engine. The implementation of online systems is further complicated when applied to jet engines, as there are severe restrictions on space, weight, and power. The systems can be further categorised into either point measurement methods or mapping methods, where point-based systems only record the temperature at a single location, but can be installed in arrays to provide more spatial information. Mapping methods allow for continuous measurement across the surface of the blade, allowing for analysis of temperature gradients.

\begin{figure}[!htbp]
    \centering
    \includegraphics[width=.8\textwidth]{./figures/tempmethodsvenn.eps}
    \caption{Temperature monitoring systems for nozzle guide vanes}\label{fig:venn}
\end{figure}

\section{Numerical methods for temperature estimation}
Numerical simulations are used to find the cooling effectiveness of turbine blades and NGVs. Results of the simulations are validated against experimental results, such as those from temperature-sensitive paints~\cite{18-guan2019enhanced}. Finite element modelling (FEM) can be used to predict the thermal load on a blade and analyse the effectiveness of blade cooling methods. There are two methods of calculations, uncoupled or coupled. Uncoupled methods are less computational intensive as the heat transfer coefficients used in the calculations are only computed at key engine operating conditions. Coupled methods use computational fluid dynamic (CFD) calculations to iterate the heat transfer coefficients, which improves accuracy over uncoupled methods. Findeisen \textit{et al.}~\cite{Findeisen2017} compared two coupled methods, FEM1D and FEM2D. The FEM1D method uses simplified models of the cooling system making it suitable for initial designs, while the FEM2D method uses three dimensional CFD simulations that are more realistic but nearly a hundred times slower than FEM1D. This makes FEM2D models more suitable for comparison with experimental data. In order to validate complex thermal models over 80\% of the surface should be measured, which is not possible with point measurement systems such as thermocouples~\cite{Peral2019}. Thermal maps allow manufacturers to better understand the effects of air flow on specific areas of components, which can help to improve the life cycle of a part. Areas of increased thermal stress can be identified, which may require additional cooling or design adjustment, while also highlighting less critical areas that could be reduced in weight or cooling. 

\section{Offline Monitoring Methods}

Offline monitoring methods rely on an irreversible physical change to occur in order to accurately record the peak temperature of a surface. A summary of the available offline methods is shown in table~\ref{tab:offline}. Thermal paints and thermal history sensors are the most commonly used methods as they allow temperature to be mapped across the surface of the blade or vane while being less intrusive than other methods.

\subsection{Thermal paint or Temperature indicating paint}\label{thermalpaint} 
Specially formulated paints can be applied to the surface of a blade or vane that irreversibly change colour after exposure to high temperatures, recording the peak temperature at which they were exposed. Irreversible thermochromism takes place because of decomposition, i.e. changes to the crystal structure because of chemical reactions that produce new compounds. The change in colour is both a function of temperature and time, requiring careful control of exposure time for accurate analysis. Yang \textit{et al.}~\cite{Yang2015a} describe the development of temperature indicating paints including their formulation, application, and test methodology.

Paints can be categorised as single-change or multi-change, whereby single-change paints change colour once after exceeding a particular temperature threshold, and multi-change paints undergo multiple colour changes as temperature increases past a number of thresholds. Multi-change paints are more suitable for use on parts with large temperature gradients, such as turbine blades or NGVs. Examples of commercially available thermal paints produced by Thermographic Measurements Ltd, UK, can be seen in figures~\ref{fig:singlepaints} and~\ref{fig:multipaints}~\cite{Neely2006}.

\begin{figure}[!htbp]
    \centering
    \includegraphics[width=.8\textwidth]{./figures/Single change paints edit.png}
    \caption{Calibration colour maps for single-change paints: SC155, SC240, SC275, SC367, SC400, SC458, SC550, SC630~\cite{Neely2006}.}\label{fig:singlepaints}
\end{figure}

\begin{figure}[!htbp]
    \centering
    \includegraphics[width=.8\textwidth]{./figures/multi-change paints edit.png}
    \caption{Calibration colour maps for multi-change paints: MC135--2, MC165--2, MC395--3, MC490--10, MC520--7~\cite{Neely2006}.}\label{fig:multipaints}
\end{figure}

Results can be visually interpreted by an operator drawing isothermal lines, or analysed with automated image processing techniques. When analysis is carried out by an operator only a small number of temperature changes can be identified, where colour changes are obvious. In order to measure a greater range of temperature changes other blades can be monitored with different paints during the same test. Results from the different blades can then be combined to provide a more detailed map. When analysis is carried out using image processing techniques accuracy can be improved to identify colour changes representing $\pm$10\si{\degreeCelsius} changes, however inconsistent lighting conditions can cause significant sources of error. A large range of different paints can be used, covering the temperature range 150\si{\degreeCelsius}--1300\si{\degreeCelsius}. This range can be further extended to 1500\si{\degreeCelsius} with the use of fluorescent paint that is analysed under UV rather than visible light~\cite{Lempereur2008}, which reduces the chance of operator error and the effect of variable lighting conditions~\cite{Coleto2006}.

Engine test data is often used to validate numerical models. In order to produce quantitative results from the use of thermal paints, the engine test data must be compared with calibration data, without which the paints only provide qualitative spatial information. Laboratory-based tests can be carried out to replicate engine conditions while the temperature is monitored using other methods (e.g.\ thermocouples, IR cameras) for validation of the real engine data. Another option is to validate the real data using predictions based on the paint's chemistry~\cite{Neely2010}.

Colour changes are cross sensitive to environmental gases which must be accounted for when comparing results to calibration charts. Low paint durability means that testing time should be limited to between 3 and 5 minutes at peak temperatures for best results.~\cite{Bird1998}. Longer exposure time makes it difficult to determine if a colour change has occurred due to peak temperatures over a short period or lower temperatures over a longer period. Dismantling the engine for analysis can be time consuming and additional tests require the paint to be removed and reapplied before the engine is reconstructed. Many thermal paints contain cobalt, nickel, or lead compounds that are restricted for use by EU legislation (REACH~\cite{EuropeanParliament2006}) because of their toxicity.

\subsection{Thermal history sensors} 
Thermal history sensors are based on ceramic materials doped with luminescent transition or rare-earth ions. They are similar in principle to thermal paints in that exposure to high temperatures causes irreversible changes to their optical properties. The method was originally proposed by Feist in 2007~\cite{Feist2011}. Resolution and accuracy is improved in comparison to thermal paints as the optical change is continuous across the surface of the part and can be analysed with an emission detector, as with online thermographic phosphors. Parts do not have to be fully dismantled for analysis, and the phosphors do not contain restricted chemicals (EU REACH regulation~\cite{EuropeanParliament2006}). Biswas and Feist~\cite{KarmakarBiswas2014} discuss the theoretical background of thermal history sensors, their analysis methods, and carry out an experimental comparison against thermal paints and thermocouples. Results show good agreement between thermal paints and thermal history coating from 400\si{\degreeCelsius} to 900\si{\degreeCelsius}. Standard deviation of the thermal history coating measurement is typically below 5\si{\degreeCelsius}. Peral \textit{et al.}~\cite{Peral2019} has developed an uncertainty model for the use of thermal history sensors from 400\si{\degreeCelsius} to 750\si{\degreeCelsius}, indicating that the maximum estimated uncertainty was $\pm6.3$\si{\degreeCelsius} or $\pm 13$\si{\degreeCelsius} for 67\% or 95\% confidence levels respectively. This is believed to be well within the uncertainty of thermal models and the requirements for temperature measurements in harsh environments on gas turbines. 

Araguas Rodriguez \textit{et al.}~\cite{AraguasRodriguez2018a} have carried out temperature measurements between 350\si{\degreeCelsius} and 900\si{\degreeCelsius} on three types of NGVs (without cooling, internally cooled, externally cooled) and compared the results to an embedded thermocouple. The THP was made up of an oxide ceramic pigment mixed into a water-based binder doped with lanthanide ions. Measurements are carried out using a handheld probe, each taking approximately five seconds. Precision of the measurement was $\pm$5\si{\degreeCelsius}, and results were within the min-max range of the thermocouple results. An extended test (around 50 hours) using THPs was carried out by Pilgrim~\cite{Pilgrim2017}. No damage to the THP was observed, and results are in agreement with temperature sensitive paints and CFD models (10\si{\degreeCelsius} difference between them). This shows an improvement over temperature sensitive paints, which can only be used for short term ($\sim$5 minutes) engine tests. The processes behind the optical changes that are caused by temperature are explained by Rabhiou \textit{et al.}~\cite{Rabhiou2011}, followed by experimental results that show a Y$_{2}$SiO$_{5}$:Tb phosphor suitable for use up to 1000\si{\degreeCelsius}, with potential for further development to extend the upper temperature limit to 1400\si{\degreeCelsius}. A resolution comparison between temperature sensitive paints, pyrometry, and thermal history sensors is given by Amiel~\cite{Amiel2018} in Table 2. Pyrometry methods provide the highest resolution (0.3\si{\degreeCelsius}), followed by thermal history sensors (1--5\si{\degreeCelsius}), and finally temperature sensitive paints (10--100\si{\degreeCelsius}). Heyes~\cite{Heyes2012} carried out tests using a Y$_2$SiO$_5$:Tb phosphor suspended in IP 600 binder that could be used for temperature measurement from 400\si{\degreeCelsius} to 900\si{\degreeCelsius}. Results showed that the use of the binder caused some signal degradation, inhibiting the crystallisation process of the phosphor. This indicated the potential benefits of integrating the phosphors into thermal barrier coatings (TBCs), which negates the need for a binder. A preliminary investigation was carried out utilising air plasma spraying (APS) of YAG:YSZ:Dy which showed promising results. 

Although accuracy and resolution are improved in comparison to thermal paints, optical access to the parts is still required for analysis, which requires dismantling of the engine but not the parts themselves. Paints need to be removed and reapplied to carry out additional tests. The application of the paints requires sophisticated coating technologies which limits their uses. Careful phosphor selection is required as emission intensity decreases with increasing temperatures~\cite{Feist2001}.

\subsection{Other offline sensors} 
A number of other offline sensors have been developed for use on turbine blades or vanes that are described below. The use of these sensors is limited in comparison to temperature sensitive paints or thermal history sensors as they only provide point measurements while being more intrusive to blade operation.

Crystal temperature sensors have been developed to more accurately measure temperature gradients across blades in comparison with thermal paint methods. A crystal structure is installed on surface of the blade using thermo-cement and exposed to neutron radiation (“illumination”), causing the atomic lattice to expand. When exposed to heat the lattice relaxes, which can be analysed with an X-ray diffraction microscope. If exposure time is also measured the temperature can be deduced from comparison with a calibration diagram. Accuracy of the method is $\pm 10$\si{\degreeCelsius}, up to at least 1400\si{\degreeCelsius}~\cite{Annerfeldt2004}.

\begin{figure}
    \centering
    \includegraphics[width=.8\textwidth]{./figures/crystal temp sens.png}
    \caption{Crystal temperature sensors installed on a NGV~\cite{Annerfeldt2004}}\label{fig:crystal}
\end{figure}

Glass ceramic arrays rely on using materials that all undergo different allotropic phase transformations when exposed to temperature changes. Using a single material will result in the same phase change whether the material is exposed to a high temperature for a short time, or a low temperature for a long time, which can be accounted for when using a number of materials. As temperature and exposure time increases optical transmittance is reduced, which can be measured using a UV light spectrophotometer. A disadvantage of this method is that it is strongly influenced by moisture in the environment (as found in aircraft engine exhausts), which causes large changes in the crystallisation structure of the glass. It also suffers from an inability to account for overheat events unless they are of sufficient magnitude (\textgreater 50\si{\degreeCelsius})~\cite{Fair2008}. 

Metallurgical temperature sensors are a family of offline sensors that can be machined into almost any shape. They are best suited to applications where wire connections are not possible, especially those where the sensor is immersed in a corrosive medium. Their accuracy is poor in comparison to thermocouples (around 10\si{\degreeCelsius}) and they can only be used to give rough estimates of temperature. Stainless steel “Feroplugs” reduce in ferromagnetism with increasing temperature exposure, which can be measured using a ferritscope or magnetic suspectometer. The reduction in ferromagnetism fully diminishes above 600\si{\degreeCelsius}, the upper limit of temperature measurement. Sigmaplugs have been developed to extend the measurement range, introducing a new phase that can be measured up to 900\si{\degreeCelsius}. Templugs are made of steel alloys that reduce in hardness with increasing temperature. The change in hardness can be measured and compared with calibration charts to determine the temperature if exposure time is known. Accuracy is around 15\si{\degreeCelsius} up to 600\si{\degreeCelsius}, reducing to 25\si{\degreeCelsius} up to 650\si{\degreeCelsius}~\cite{Lo2012}. Madison \textit{et al.}~\cite{Madison2013} compared metallurgical temperature sensors with thermocouples when measuring piston temperatures in a running engine. Results were comparable however the metallurgical sensors were heavily influenced by areas with large thermal gradients.

\section{Online Monitoring Methods}

Online methods are used to measure temperature in real-time. This is particularly useful in comparison to offline methods as the monitoring can take place at start-up and cool down of the engine, rather than only providing a peak result. Another benefit over offline methods is the reduction in research and development time as the engine does not need to be dismantled between each test, which in turn reduces cost. This enables in-service data collection as opposed to short-term engine test data gathered from offline monitoring. Additional data helps to further validate numerical models when online systems are installed in test rigs, but they can also be used for condition monitoring applications on in-service engines. More comprehensive monitoring is beneficial in reducing maintenance schedules as these can be based on component health rather than a fixed number of operational hours. This has the benefits of reducing maintenance costs whilst minimising downtime. A reduction in the frequency of blade replacement is also beneficial to the environment. Table~\ref{tab:online} shows the currently available online monitoring methods. Thermographic phosphors are the most commonly used method as they are less intrusive than point-based thermocouple or RTD arrays, while being more reliable than pyrometry methods.

\subsection{Thermocouples} 
Conventional thermocouples are not suitable for use on turbine blades or NGVs, as their large size disrupts airflow, and reliably attaching them to the surface of a blade is difficult. Thin film thermocouples (TFTC) on the other hand can be directly fabricated on to the surface of a turbine blade or NGV, greatly improving long term durability in comparison to conventional thermocouples. COMSOL simulations carried out by Duan \textit{et al.}~\cite{Duan2020} show a ${\sim}$170\si{\degreeCelsius} difference between conventional thermocouples and their thin-film counterparts due to the distance between the thermocouple joint and the blade surface. The placement of a conventional thermocouple in its insulation filling adds an additional uncertainty of around $500\times500$ \si{\um}, which has a temperature difference from top to bottom of ${\sim}$300\si{\degreeCelsius}. The uncertainty associated with the use of conventional thermocouples shows the motivation for the development of thin film equivalents.

\begin{figure}
    \centering
    \includegraphics[width=.8\textwidth]{./figures/tftcNGV.png}
    \caption{TFTC array installed on a NGV~\cite{Duan2020}}\label{fig:tftcngv}
\end{figure}

The small footprint of TFTCs does not affect airflow, leading to more accurate measurements of fast temperature fluctuations. TFTCs have an extremely fast response time (\textless 1 \si{\us}), are low cost, and have very fine spatial resolution at a single point. S-type platinum plus platinum 10\% rhodium thermocouples have been shown to be stable up to 1100\si{\degreeCelsius}. However, oxidation of rhodium (which would occur in the gas path of a turbine) causes the thermoelectric potential to decrease, which in turn limits the upper working temperature of these types of thermocouple~\cite{Kreider1993}. The main challenges to overcome with TFTCs are: developing an electrical insulation film that can withstand the harsh conditions of the turbine (high temperatures, shock, and vibration), and applying the sensor to curved surfaces.

Yang \textit{et al.}~\cite{Yang2020} have shown that an S-type thermocouple can be fabricated using pulsed laser deposition (PLD) techniques that functions up to 700\si{\degreeCelsius}. To improve thermal stability at high temperatures composite ceramic oxides can be used, which have good anti-oxidation properties and have been shown to maintain high thermal outputs up to 1200\si{\degreeCelsius}~\cite{Liu2018}. Direct-Write Thermal Spray (DWTS) is a process by which powder, wire, or rods can be adhered to a complex substrate such as a blade or vane. A thermocouple can be embedded into the thermal barrier coating (TBC) on the surface of an NGV in this way and has been shown to perform well up to 450\si{\degreeCelsius}~\cite{Zhang2018b}. The addition of Al$_2$O$_3$ to YSZ(yttria-stabilized zirconia)-based TBCs improves their electrical insulation above 400\si{\degreeCelsius} which allows TFTCs to be used effectively~\cite{Gao2016}. A summary of THTC systems is given by Satish \textit{et al.}~\cite{Satish2017} along with the implementation of a K-type TFTC on an NGV deposited using an E-beam evaporation technique. Two methods of measuring temperature distribution on a TBC coating are given by Z. Ji \textit{et al.}~\cite{Ji2019} whereby Pt soldering dots are placed along a PtRH thin film. Measurements are possible up to 1200\si{\degreeCelsius} with this method.

Despite the improvement of thin film thermocouples over conventional thermocouples they are still an invasive monitoring method. Only point measurements are possible unless installed in dense arrays which can have an impact on blade operation, requiring complex wiring systems. The connection between thin-film and lead wires is the primary failure point when installed on turbine blades~\cite{Lei1997}. Oxidation has a significant impact on the operation of TFTCs at high temperatures, causing drift and delamination~\cite{Guk2016}, which makes the use of an insulation material vital.

\subsection{Pyrometry} 
Optical pyrometry is a non-contact method of measuring thermal radiation emitted from the surface of a material. There is no upper temperature limit as thermal radiation increases with temperature, but there is a lower limit of around 500\si{\degreeCelsius}~\cite{Kerr2004}. Measurement response is fast in comparison to conventional thermocouples as there is no thermal inertia to overcome. In order to implement a pyrometry system in a turbine optical access is required, which can be achieved by routing a probe through the wall of the turbine casing and connecting it to a detector via fibre optic cable, away from the high temperature environment of the turbine~\cite{Kerr2004}. The system can be considered a point measurement system as the probe is focused on an area 1--10 mm in diameter however the point can be scanned across the surface to produce temperature maps~\cite{Li2019}. A review of fibre optic thermometry methods is given by Yu \textit{et al.}~\cite{Yu2009a}, covering blackbody, infrared, and fluorescence optical thermometers. For accurate results the emittance of the blades must be measured prior to installation to later be used in correcting the output of the pyrometer, however the emittance is also affected by factors such as surface roughness, oxidation state, and chemical composition, which will affect the overall accuracy of the system. A number of solutions have been proposed to improve the accuracy of emittance measurement, namely dual and triple spectral pyrometry~\cite{Kerr2002, Li2018}. A review of multi-spectral pyrometry systems is given by Araújo \textit{et al.}~\cite{Araujo2017}. Accuracy is also affected by reflected radiation from other surfaces within the turbine, absorption of radiation in the gas path, interference from hot particles, and deposits forming on the surface of the probe lens. The effectiveness of pyrometry is negatively affected by the application of thermal barrier coatings (TBCs), which reduces emittance while increasing reflectivity of the blades. The thermal gradient of TBCs ($\sim$0.5\si{\degreeCelsius}/\si{\um}) means that careful wavelength selection is required to ensure that measurement is taking place at the required depth~\cite{Gentleman2006}. A number of authors have tested pyrometers on turbine engine test rigs~\cite{Becker1994,Frank2001,Wang2015}. Taniguchi \textit{et al.}~\cite{Taniguchi2006} have carried out temperature measurements using a pyrometer on a turbine blade rotating at 14,000 RPM. The system has a measurement range of 700--1150\si{\degreeCelsius} with an accuracy of $\pm$3\si{\degreeCelsius}, over a 2 mm diameter area. The measurement probe is only inserted into the hot section of the engine for a short time (10s) before being removed to avoid damage, which limits the long term use of this method. Infrared thermography has also been used on gas turbines for crack and defect detection rather than absolute temperature measurement~\cite{Hellberg}. Although pyrometry can provide high accuracy (in favourable conditions) and fast response times, the need for optical access and their large size restricts their uses to dedicated test rigs and large gas turbines. It is unlikely that a pyrometry system could be installed inside of a jet engine for SHM applications without considerable reductions in size and weight. During long term use the build up of deposits on the surface of the probe lens will reduce accuracy and require regular maintenance. In order to produce temperature maps the probe must be scanned across the surface which would require mechanical movement of the probe, a potential point of failure.

\begin{figure}
    \centering
    \includegraphics[width=.8\textwidth]{./figures/kerrpyro.eps}
    \caption{Cross sectional view of pyrometer installation~\cite{Kerr2004}}\label{fig:crosspyro}
\end{figure}

\begin{figure}
    \centering
    \includegraphics[width=.8\textwidth]{./figures/siemens IR edit-01.png}
    \caption{IR camera in use on a Siemens SGT-750 gas turbine~\cite{Hellberg}}\label{fig:IRsiemens}
\end{figure}

\subsection{Temperature sensitive paints} 
Temperature sensitive paints (TSPs) differ from thermal paints or temperature indicating paints in that they do not undergo permanent changes to their structure after exposure to high temperatures. The temperature dependence of luminescent molecules can be measured after application to the surface of a material using a polymer binder. The luminophore is excited by a light source, causing thermal quenching to take place. The intensity or decay lifetime of the luminescence can be measured using a camera, both of which are temperature dependent~\cite{Liu2005}. A selection of TSPs are described by Patel \textit{et al.}~\cite{Patel2016}, showing their useful temperature ranges. At low temperatures (\textless 200\si{\degreeCelsius}) metal-ligand complexes (MLCs) are used as luminophores, whereas at high temperature thermographic phosphors are used.

\subsection{Thermographic phosphors}\label{Thermographic phosphors}
Thermographic phosphors can be used in a very similar way to the metal-ligand complexes (MLCs) used in temperature sensitive paints however they are suitable for use up to much higher temperatures (${\sim}$1600\si{\degreeCelsius}). They can be excited and analysed in the same way as TSPs using laser excitation and fluorescence detectors. A number of authors have carried out reviews of thermographic phosphor systems, covering the principles and theory of fluorescence~\cite{Khalid2008a}, instrumentation~\cite{Allison1997}, implementation~\cite{Khalid2010}, and discussion of phosphor materials and error analysis~\cite{Brubach2013}. Phosphors can be applied to a blade or vane in a number of ways, the simplest of which is to mix the phosphor with a binder and apply it to the surface of the TBC, which provides a surface temperature measurement. A phosphor layer can also be applied beneath the TBC, which allows for monitoring of the temperature critical bonding layer. This is difficult to achieve as YSZ is mostly opaque to UV radiation, making excitation of the phosphor a challenge~\cite{AbouNada2016}. Phosphors can also be directly integrated into thermal barrier coatings (TBCs), acting as thermal insulation and as a sensor simultaneously. The first investigation into using phosphors embedded into thermal barrier coatings (TBCs), called ‘smart coating’ or ‘thermal barrier sensor coating’ was carried out by Feist~\cite{Feist2001}. Further development has shown that the addition of phosphors to produce sensor coatings does not affect the structure of TBCs and they can be applied using the same air plasma spray (APS) process as non-doped TBCs~\cite{Chen2005a}. The system has been applied to an NGV in a Rolls-Royce Viper 201 engine with optical access provided through a dedicated window~\cite{Feist2013}. A comparison between temperature measurements from a TBC doped sensor, pyrometer, and thermocouple shows a large temperature gradient between the surface of the TBC and the surface of the substrate, with the phosphor sensing taking place close to the bond coat interface~\cite{YanezGonzalez2015}.

Online temperature measurements from 513\si{\degreeCelsius} to 767\si{\degreeCelsius} have been carried out on an operating engine by Jenkins \textit{et al.}~\cite{Jenkins2020}. Measurements were carried out at rotational speeds up to 32,750 RPM. A laser was used for excitation of two phosphors applied to the TBC of the turbine blades, while a fibre optic probe was used for detection. Results were found to be within 25\si{\degreeCelsius} of of those estimated by the engine manufacturer. It is suggested that measurement range (from engine off to maximum engine power) can be improved by utilising Y$_{2}$O$_{3}$:Er phosphors~\cite{Eldridge2019}. Nau \textit{et al.}~\cite{Nau2019} have presented results from the use of five different phosphors that allow measurements from room temperature up to 1500\si{\degreeCelsius}. Comparison of the phosphors show that their individual temperature ranges are limited (400\si{\degreeCelsius}). Measurements were carried out on two model combustor systems, and at high pressure, to demonstrate the potential application in real turbines. Measurements are demonstrated with a high speed camera, allowing for temporal resolution up to 1 kHz.

\begin{figure}
    \centering
    \includegraphics[width=.8\textwidth]{./figures/yaneztempTBC.eps}
    \caption{A comparison of online monitoring systems (reproduced from Yañez Gonzalez \textit{et al.}~\cite{YanezGonzalez2015})}\label{fig:yaneztemp}
\end{figure}

The main limitation of using thermographic phosphors is the need for optical access, both for excitation and detection, which is normally achieved with fibre optic probes. Careful selection of phosphor material is required to cover the entire temperature range of interest and to ensure that emission intensity is high enough for accurate detection at elevated temperatures~\cite{Feist2001}. In the case of turbine blades the excitation and detection needs to be synchronised with blade rotation which requires extremely fast response times~\cite{Gentleman2006}, and is limited by the decay rate of the phosphor.  

\subsection{Other methods} 
Duan \textit{et al.}~\cite{Duan2018} have found that the electrically conductive nature of TBCs at elevated temperatures (above 600\si{\degreeCelsius}) can be utilised as a form of smart sensor, by measuring a change in resistance with temperature. Experimental data shows good repeatability up to 950\si{\degreeCelsius}, measurement error of less than 3\%, fast response time (${\sim}$1s), and stability comparable to conventional thermocouples and TFTCs.

\newgeometry{margin=3cm}  
\begin{landscape}

\begin{table*}[p]
        \resizebox{\linewidth}{!}{%
        \begin{tabular}{@{}llllll@{}}
        \toprule
        \textbf{Method} &   \textbf{\begin{tabular}[c]{@{}l@{}}Temperature \\ range (\si{\degreeCelsius})\end{tabular}} &
          \textbf{\begin{tabular}[c]{@{}l@{}}Temperature \\ resolution (\si{\degreeCelsius})\end{tabular}} &
          \textbf{\begin{tabular}[c]{@{}l@{}}Spatial \\ resolution\end{tabular}} &
            \textbf{Sensor type} &
            \textbf{Interrogation method} \\ \hline
        Crystal   temperature sensors & up to 1400 & $\pm$ 10     & Low                 & Surface mounted                  & Illumination + observation                \\
        Templugs                      & 650        & 15--25     & Low (single points) & Embedded in substrate            & Analysis of ferromagnetism/hardness/phase \\
        Glass   ceramic arrays        & 1100       & Not given & Low                 & Surface mounted                  & Observation only                          \\
        Thermal   paints (TIP)        & Up to 1500 & 10--100    & Moderate            & Surface layer                    & Observation only                          \\
        Thermal   history sensors     & 150--1400   & $\pm$ 5      & High                & Surface layer or embedded in TBC & Illumination + observation                \\ \bottomrule
        \end{tabular}%
        }
        \caption{\label{tab:offline} Summary of offline temperature monitoring methods}
\end{table*}

\begin{table*}[p]
    \resizebox{\linewidth}{!}{%
    \begin{tabular}{@{}llllll@{}}
    \toprule
    \textbf{Method} & \textbf{\begin{tabular}[c]{@{}l@{}}Temperature \\ range (\si{\degreeCelsius})\end{tabular}} & 
    \textbf{\begin{tabular}[c]{@{}l@{}}Temperature \\ resolution (\si{\degreeCelsius})\end{tabular}} & 
    \textbf{Spatial resolution} & \textbf{Sensor type} & \textbf{Interrogation method} \\ \midrule
    Thin film thermocouples (TFTCs)     & Up to 1200             & $\pm$ 2 & Low (can be installed in arrays) & Surface mounted                  & Electrical connection      \\
    Infrared Thermography/Pyrometry     & 500-\textgreater{}2000 & 0.3  & Low (can be scanned)             & Optical probe                    & Observation only           \\
    Temperature sensitive paints (MLCs) & \textless{}200         & 1--5  & High                             & Surface layer                    & Illumination + observation \\
    Thermographic phosphors             & Up to 1600             & 1--5  & High                             & Surface layer or embedded in TBC & Illumination + observation \\ \bottomrule
    \end{tabular}%
    }
    \caption{\label{tab:online} Summary of online temperature monitoring methods}
\end{table*}
\end{landscape}
\clearpage
\restoregeometry