\chapter{Research summary \& future plans}\label{plan}

\section{Research summary}

Prior to the 18 month confirmation review the following research has been undertaken:

\begin{itemize}
    \item Evaluation of the currently used methods for temperature monitoring of NGVs/turbine blades, with a focus on online methods. The outcome of this study can provide the requirements for the development of a new system. This is addressed in Section~\ref{tempmonitoringmethods}.
    \item An investigation into whether ultrasonic methods are a suitable alternative to the currently available monitoring systems, based on a number of factors such as: response time, accuracy, reliability, power consumption, and possible sensor installation methods. This is addressed in Section~\ref{ultrasonicmonitoring}. This study contributes to answering research question~\ref{itm:1}.
    \item Identifying a method of exciting Lamb waves in plate-like structures. This is addressed in Section~\ref{wavegen}. This study contributes to answering research question~\ref{itm:2}.
    \item Understanding the effect of temperature on Lamb wave propagation. This is addressed in Section~\ref{lambwavesensing}. This study contributes to answering research question~\ref{itm:1}.
    \item Verification of the theoretical temperature sensitivities of Lamb wave modes with experimental data. Measurements up to 100\si{\degreeCelsius} in an aluminium plate. This is addressed in Section~\ref{theory}. This study contributes to answering research question~\ref{itm:2}.
    \item COMSOL modelling of the experimental test system. The model will be used in future studies. This is addressed in Section~\ref{simulations}. This study contributes to answering research questions~\ref{itm:3},~\ref{itm:4},~\ref{itm:5}, and~\ref{itm:6}.
\end{itemize} 

%Considering the outcome of the experimental study it has been shown that temperature can be monitored using ultrasound techniques. The next stage of the study is to investigate the effect of cooling holes on the propagation of waves, and to explore the possibility of monitoring temperature in multiple locations by analysing the reflections from a number of holes.

%In order to adapt the system to nozzle guide vanes a different transducer configuration is required that can be permanently coupled to the structure, and can operate at considerably higher temperatures than the current system. As shown previously the most suitable candidate for achieving this are waveguides that can be used to distance the transducers from the high temperature environment. A transducer configuration that utilises waveguides will be modelled using COMSOL, and the suitability of the system for use on NGVs will be evaluated. Based on the outcome of this study, and time permitting, a waveguide system will be tested experimentally, in order to validate the model. %

\newpage
\section{Future plans}

\begin{itemize}
    \item The effect of plate holes on Lamb wave mode propagation will be investigated through simulation and experimentation. This will mimic the cooling hole structure found on NGVs. The effect on amplitude and mode conversion will be investigated for a number of modes, to determine the most suitable mode for temperature monitoring. This study will partially answer research questions~\ref{itm:3} and~\ref{itm:4}.
    \item The ability to monitor temperature at a number of locations through the analysis of acoustic reflections will be explored. This will follow on from the previous study, focusing on the use of the mode previously determined to be most suitable. Both pulse-echo and pitch-catch transducer configurations will be investigated. This study will answer research question~\ref{itm:5}, and contribute to answering research question~\ref{itm:6}.
    \item The effect of thermal barrier coatings (TBCs) on wave propagation will be investigated using COMSOL models. This study will partially answer research question~\ref{itm:3}. 
    \item To operate at higher temperatures a new transducer configuration will be required. The most suitable option is to distance the transducers away from the high temperatures using waveguides, coupling into the test structure using a system of Hertzian contact points. COMSOL simulations will be used to investigate the potential of this method. This study will partially answer research question~\ref{itm:6}. 
    \item Depending on the outcome of the previous COMSOL study and time constraints, a waveguide system will be tested experimentally. This study will partially answer research question~\ref{itm:6}. 
\end{itemize}

\section{Gantt chart}

\includepdf[landscape=true,fitpaper=true]{./figures/gantt.pdf}